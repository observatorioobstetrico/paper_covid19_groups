% Options for packages loaded elsewhere
\PassOptionsToPackage{unicode}{hyperref}
\PassOptionsToPackage{hyphens}{url}
%
\documentclass[
]{article}
\usepackage{lmodern}
\usepackage{amssymb,amsmath}
\usepackage{ifxetex,ifluatex}
\ifnum 0\ifxetex 1\fi\ifluatex 1\fi=0 % if pdftex
  \usepackage[T1]{fontenc}
  \usepackage[utf8]{inputenc}
  \usepackage{textcomp} % provide euro and other symbols
\else % if luatex or xetex
  \usepackage{unicode-math}
  \defaultfontfeatures{Scale=MatchLowercase}
  \defaultfontfeatures[\rmfamily]{Ligatures=TeX,Scale=1}
\fi
% Use upquote if available, for straight quotes in verbatim environments
\IfFileExists{upquote.sty}{\usepackage{upquote}}{}
\IfFileExists{microtype.sty}{% use microtype if available
  \usepackage[]{microtype}
  \UseMicrotypeSet[protrusion]{basicmath} % disable protrusion for tt fonts
}{}
\makeatletter
\@ifundefined{KOMAClassName}{% if non-KOMA class
  \IfFileExists{parskip.sty}{%
    \usepackage{parskip}
  }{% else
    \setlength{\parindent}{0pt}
    \setlength{\parskip}{6pt plus 2pt minus 1pt}}
}{% if KOMA class
  \KOMAoptions{parskip=half}}
\makeatother
\usepackage{xcolor}
\IfFileExists{xurl.sty}{\usepackage{xurl}}{} % add URL line breaks if available
\IfFileExists{bookmark.sty}{\usepackage{bookmark}}{\usepackage{hyperref}}
\hypersetup{
  pdftitle={COVID-19 outcomes in hospitalized puerperal, pregnant, and neither pregnant nor puerperal women: a population study},
  pdfauthor={Codes and outputs of Propensity Score Matching},
  hidelinks,
  pdfcreator={LaTeX via pandoc}}
\urlstyle{same} % disable monospaced font for URLs
\usepackage[margin=1in]{geometry}
\usepackage{color}
\usepackage{fancyvrb}
\newcommand{\VerbBar}{|}
\newcommand{\VERB}{\Verb[commandchars=\\\{\}]}
\DefineVerbatimEnvironment{Highlighting}{Verbatim}{commandchars=\\\{\}}
% Add ',fontsize=\small' for more characters per line
\usepackage{framed}
\definecolor{shadecolor}{RGB}{248,248,248}
\newenvironment{Shaded}{\begin{snugshade}}{\end{snugshade}}
\newcommand{\AlertTok}[1]{\textcolor[rgb]{0.94,0.16,0.16}{#1}}
\newcommand{\AnnotationTok}[1]{\textcolor[rgb]{0.56,0.35,0.01}{\textbf{\textit{#1}}}}
\newcommand{\AttributeTok}[1]{\textcolor[rgb]{0.77,0.63,0.00}{#1}}
\newcommand{\BaseNTok}[1]{\textcolor[rgb]{0.00,0.00,0.81}{#1}}
\newcommand{\BuiltInTok}[1]{#1}
\newcommand{\CharTok}[1]{\textcolor[rgb]{0.31,0.60,0.02}{#1}}
\newcommand{\CommentTok}[1]{\textcolor[rgb]{0.56,0.35,0.01}{\textit{#1}}}
\newcommand{\CommentVarTok}[1]{\textcolor[rgb]{0.56,0.35,0.01}{\textbf{\textit{#1}}}}
\newcommand{\ConstantTok}[1]{\textcolor[rgb]{0.00,0.00,0.00}{#1}}
\newcommand{\ControlFlowTok}[1]{\textcolor[rgb]{0.13,0.29,0.53}{\textbf{#1}}}
\newcommand{\DataTypeTok}[1]{\textcolor[rgb]{0.13,0.29,0.53}{#1}}
\newcommand{\DecValTok}[1]{\textcolor[rgb]{0.00,0.00,0.81}{#1}}
\newcommand{\DocumentationTok}[1]{\textcolor[rgb]{0.56,0.35,0.01}{\textbf{\textit{#1}}}}
\newcommand{\ErrorTok}[1]{\textcolor[rgb]{0.64,0.00,0.00}{\textbf{#1}}}
\newcommand{\ExtensionTok}[1]{#1}
\newcommand{\FloatTok}[1]{\textcolor[rgb]{0.00,0.00,0.81}{#1}}
\newcommand{\FunctionTok}[1]{\textcolor[rgb]{0.00,0.00,0.00}{#1}}
\newcommand{\ImportTok}[1]{#1}
\newcommand{\InformationTok}[1]{\textcolor[rgb]{0.56,0.35,0.01}{\textbf{\textit{#1}}}}
\newcommand{\KeywordTok}[1]{\textcolor[rgb]{0.13,0.29,0.53}{\textbf{#1}}}
\newcommand{\NormalTok}[1]{#1}
\newcommand{\OperatorTok}[1]{\textcolor[rgb]{0.81,0.36,0.00}{\textbf{#1}}}
\newcommand{\OtherTok}[1]{\textcolor[rgb]{0.56,0.35,0.01}{#1}}
\newcommand{\PreprocessorTok}[1]{\textcolor[rgb]{0.56,0.35,0.01}{\textit{#1}}}
\newcommand{\RegionMarkerTok}[1]{#1}
\newcommand{\SpecialCharTok}[1]{\textcolor[rgb]{0.00,0.00,0.00}{#1}}
\newcommand{\SpecialStringTok}[1]{\textcolor[rgb]{0.31,0.60,0.02}{#1}}
\newcommand{\StringTok}[1]{\textcolor[rgb]{0.31,0.60,0.02}{#1}}
\newcommand{\VariableTok}[1]{\textcolor[rgb]{0.00,0.00,0.00}{#1}}
\newcommand{\VerbatimStringTok}[1]{\textcolor[rgb]{0.31,0.60,0.02}{#1}}
\newcommand{\WarningTok}[1]{\textcolor[rgb]{0.56,0.35,0.01}{\textbf{\textit{#1}}}}
\usepackage{graphicx,grffile}
\makeatletter
\def\maxwidth{\ifdim\Gin@nat@width>\linewidth\linewidth\else\Gin@nat@width\fi}
\def\maxheight{\ifdim\Gin@nat@height>\textheight\textheight\else\Gin@nat@height\fi}
\makeatother
% Scale images if necessary, so that they will not overflow the page
% margins by default, and it is still possible to overwrite the defaults
% using explicit options in \includegraphics[width, height, ...]{}
\setkeys{Gin}{width=\maxwidth,height=\maxheight,keepaspectratio}
% Set default figure placement to htbp
\makeatletter
\def\fps@figure{htbp}
\makeatother
\setlength{\emergencystretch}{3em} % prevent overfull lines
\providecommand{\tightlist}{%
  \setlength{\itemsep}{0pt}\setlength{\parskip}{0pt}}
\setcounter{secnumdepth}{-\maxdimen} % remove section numbering
\usepackage{booktabs}
\usepackage{longtable}
\usepackage{array}
\usepackage{multirow}
\usepackage{wrapfig}
\usepackage{float}
\usepackage{colortbl}
\usepackage{pdflscape}
\usepackage{tabu}
\usepackage{threeparttable}
\usepackage{threeparttablex}
\usepackage[normalem]{ulem}
\usepackage{makecell}
\usepackage{xcolor}

\title{COVID-19 outcomes in hospitalized puerperal, pregnant, and neither
pregnant nor puerperal women: a population study}
\author{Codes and outputs of Propensity Score Matching}
\date{08/26/2021}

\begin{document}
\maketitle

\hypertarget{description}{%
\section{1. Description}\label{description}}

This file presents the documentation of the analysis of Propensity
Scoring Method (PSM) of the article ``COVID-19 outcomes in hospitalized
puerperal, pregnant, and neither pregnant nor puerperal women: a
population study'' with authors Fabiano Elisei Serra, Rossana Pulcineli
Vieira Francisco, Patricia de Rossi, Maria de Lourdes Brizot, and Agatha
Sacramento Rodrigues.

\hypertarget{r-packages-used-functions-and-dataset-import}{%
\section{2. R packages used, functions and dataset
import}\label{r-packages-used-functions-and-dataset-import}}

The data are analyzed using the free-software R
(\url{https://www.R-project.org}) in version 4.0.3. Next, we present and
load the libraries used in the data analysis process.

\begin{Shaded}
\begin{Highlighting}[]
\CommentTok{#load packages}
\NormalTok{loadlibrary <-}\StringTok{ }\ControlFlowTok{function}\NormalTok{(x) \{}
  \ControlFlowTok{if}\NormalTok{ (}\OperatorTok{!}\KeywordTok{require}\NormalTok{(x, }\DataTypeTok{character.only =} \OtherTok{TRUE}\NormalTok{)) \{}
    \KeywordTok{install.packages}\NormalTok{(x, }\DataTypeTok{dependencies =}\NormalTok{ T)}
    \ControlFlowTok{if}\NormalTok{ (}\OperatorTok{!}\KeywordTok{require}\NormalTok{(x, }\DataTypeTok{character.only =} \OtherTok{TRUE}\NormalTok{))}
      \KeywordTok{stop}\NormalTok{(}\StringTok{"Package not found"}\NormalTok{)}
\NormalTok{  \}}
\NormalTok{\}}

\NormalTok{packages <-}
\StringTok{  }\KeywordTok{c}\NormalTok{(}
    \StringTok{"readr"}\NormalTok{,}
    \StringTok{"magrittr"}\NormalTok{,}
    \StringTok{"dplyr"}\NormalTok{,}
    \StringTok{"stringr"}\NormalTok{,}
    \StringTok{"questionr"}\NormalTok{,}
    \StringTok{"knitr"}\NormalTok{,}
    \StringTok{"forcats"}\NormalTok{,}
    \StringTok{"lubridate"}\NormalTok{,}
    \StringTok{"summarytools"}\NormalTok{,}
    \StringTok{"modelsummary"}\NormalTok{,}
    \StringTok{"kableExtra"}\NormalTok{,}
    \StringTok{"epitools"}\NormalTok{,}
    \StringTok{"WeightIt"}\NormalTok{,}
    \StringTok{"jtools"}\NormalTok{,}
    \StringTok{"survey"}\NormalTok{,}
    \StringTok{"cobalt"}\NormalTok{,}
    \StringTok{"nnet"}
\NormalTok{  )}
\KeywordTok{lapply}\NormalTok{(packages, loadlibrary)}
\end{Highlighting}
\end{Shaded}

The Influenza Epidemiological Surveillance Information System,
SIVEP-Gripe (Sistema de Informação de Vigilância Epidemiológica da
Gripe), is a nationwide surveillance database used to monitor severe
acute respiratory infections in Brazil.

Notification is mandatory for Influenza Syndrome (characterized by at
least two of the following signs and symptoms: fever, even if referred,
chills, sore throat, headache, cough, runny nose, olfactory or taste
disorders) and who has dyspnea/respiratory discomfort or persistent
pressure in the chest or O2 saturation less than 95\% in room air or
bluish color of the lips or face. Asymptomatic individuals with
laboratory confirmation by molecular biology or immunological
examination for COVID-19 infection are also reported.

For notifications in Sivep-Gripe, hospitalized cases in both public and
private hospitals and all deaths due to severe acute respiratory
infections regardless of hospitalization must be considered.

The analyzed period comprised data from epidemiological weeks 1 to 53 of
2020 (12/29/2019 - 01/02/2021) with the database downloaded on
01/11/2021 on the site
\url{https://opendatasus.saude.gov.br/dataset/bd-srag-2020}. The data
are loaded below:

\begin{Shaded}
\begin{Highlighting}[]
\CommentTok{#loading the dataset}
\NormalTok{data_all <-}\StringTok{ }\NormalTok{readr}\OperatorTok{::}\KeywordTok{read_delim}\NormalTok{(}
  \StringTok{"INFLUD-11-01-2021.csv"}\NormalTok{,}
  \StringTok{";"}\NormalTok{,}
  \DataTypeTok{escape_double =} \OtherTok{FALSE}\NormalTok{,}
  \DataTypeTok{locale =} \KeywordTok{locale}\NormalTok{(}\DataTypeTok{encoding =} \StringTok{"ISO-8859-2"}\NormalTok{),}
  \DataTypeTok{trim_ws =} \OtherTok{TRUE}
\NormalTok{)}
\end{Highlighting}
\end{Shaded}

There are 1136681 cases in the complete dataset. The case selection is
presented in the following according to the flowchart presented in the
article.

\hypertarget{case-selection-and-data-treatment}{%
\section{3. Case selection and data
treatment}\label{case-selection-and-data-treatment}}

The first filter consists of selecting the hospitalized cases. For that,
the \texttt{HOSPITAL} variable is considered, in which 1-Yes, 2-No, and
9-Ignored.

\begin{Shaded}
\begin{Highlighting}[]
\CommentTok{#Selecting only hospitalization cases}
\NormalTok{data1 <-}\StringTok{ }\NormalTok{dplyr}\OperatorTok{::}\KeywordTok{filter}\NormalTok{(data_all, HOSPITAL }\OperatorTok{==}\StringTok{ }\DecValTok{1}\NormalTok{)}
\end{Highlighting}
\end{Shaded}

When considering only confirmed hospitalized cases, we get 1061254
observations.

The second filtering consists of the cases classified as COVID-19 in the
database. The variable indicating the classification is
\texttt{CLASSI\_FIN}, with the following categories: 1-SRAG by
influenza, 2-SRAG by another respiratory virus, 3-SRAG by another
etiological agent, 4-SRAG not specified, and 5-SRAG by COVID-19.

\begin{Shaded}
\begin{Highlighting}[]
\NormalTok{questionr}\OperatorTok{::}\KeywordTok{freq}\NormalTok{(}
\NormalTok{  data1}\OperatorTok{$}\NormalTok{CLASSI_FIN,}
  \DataTypeTok{cum =} \OtherTok{FALSE}\NormalTok{,}
  \DataTypeTok{total =} \OtherTok{TRUE}\NormalTok{,}
  \DataTypeTok{na.last =} \OtherTok{FALSE}\NormalTok{,}
  \DataTypeTok{valid =} \OtherTok{FALSE}
\NormalTok{) }\OperatorTok
\StringTok{  }\NormalTok{knitr}\OperatorTok{::}\KeywordTok{kable}\NormalTok{(}\DataTypeTok{caption =} \StringTok{"Frequency table for case classification"}\NormalTok{, }\DataTypeTok{digits =} \DecValTok{2}\NormalTok{) }\OperatorTok
\StringTok{  }\KeywordTok{kable_styling}\NormalTok{(}\DataTypeTok{latex_options =} \StringTok{"hold_position"}\NormalTok{)}
\end{Highlighting}
\end{Shaded}

\begin{table}[!h]

\caption{\label{tab:unnamed-chunk-3}Frequency table for case classification}
\centering
\begin{tabular}[t]{l|r|r}
\hline
  & n & \%\\
\hline
1 & 2507 & 0.2\\
\hline
2 & 4137 & 0.4\\
\hline
3 & 2929 & 0.3\\
\hline
4 & 365992 & 34.5\\
\hline
5 & 588711 & 55.5\\
\hline
NA & 96978 & 9.1\\
\hline
Total & 1061254 & 100.0\\
\hline
\end{tabular}
\end{table}

\begin{Shaded}
\begin{Highlighting}[]
\CommentTok{#Filtering COVID-19 cases}
\NormalTok{data2 <-}\StringTok{ }\NormalTok{dplyr}\OperatorTok{::}\KeywordTok{filter}\NormalTok{(data1, CLASSI_FIN }\OperatorTok{==}\StringTok{ }\DecValTok{5}\NormalTok{)}
\end{Highlighting}
\end{Shaded}

There are 588711 selected cases for now.

Only cases of COVID-19 confirmed by RT-PCR are selected. The selection
is made as follows:

\begin{Shaded}
\begin{Highlighting}[]
\CommentTok{#Selecting COVID-19 confirmed by RT-PCR}
\NormalTok{data3 <-}\StringTok{ }\NormalTok{data2 }\OperatorTok
\StringTok{  }\NormalTok{dplyr}\OperatorTok{::}\KeywordTok{filter}\NormalTok{((PCR_SARS2 }\OperatorTok{==}\StringTok{ }\DecValTok{1}\NormalTok{) }\OperatorTok{|}
\StringTok{                  }\NormalTok{(}
\NormalTok{                    stringr}\OperatorTok{::}\KeywordTok{str_detect}\NormalTok{(DS_PCR_OUT, }\StringTok{"SARS|COVID|COV|CORONA|CIVID"}\NormalTok{) }\OperatorTok{&}
\StringTok{                      }\OperatorTok{!}\NormalTok{stringr}\OperatorTok{::}\KeywordTok{str_detect}\NormalTok{(DS_PCR_OUT, }\StringTok{"63|43|229|HK|RINO|SINCI|PARE"}\NormalTok{)}
\NormalTok{                  ) }\OperatorTok{|}
\StringTok{                  }\NormalTok{(}
\NormalTok{                    PCR_RESUL }\OperatorTok{==}\StringTok{ }\DecValTok{1} \OperatorTok{&}
\StringTok{                      }\NormalTok{CRITERIO }\OperatorTok{==}\StringTok{ }\DecValTok{1} \OperatorTok{&}
\StringTok{                      }\KeywordTok{is.na}\NormalTok{(DS_PCR_OUT) }\OperatorTok{&}
\StringTok{                      }\NormalTok{(PCR_RINO  }\OperatorTok{!=}\StringTok{ }\DecValTok{1} \OperatorTok{|}
\StringTok{                         }\KeywordTok{is.na}\NormalTok{(PCR_RINO)) }\OperatorTok{&}
\StringTok{                      }\NormalTok{(POS_PCRFLU }\OperatorTok{!=}\StringTok{ }\DecValTok{1} \OperatorTok{|}\StringTok{ }\KeywordTok{is.na}\NormalTok{(POS_PCRFLU)) }\OperatorTok{&}
\StringTok{                      }\NormalTok{(PCR_OUTRO }\OperatorTok{!=}\StringTok{ }\DecValTok{1} \OperatorTok{|}\StringTok{ }\KeywordTok{is.na}\NormalTok{(PCR_OUTRO)) }\OperatorTok{&}
\StringTok{                      }\NormalTok{(POS_PCROUT }\OperatorTok{!=}\StringTok{ }\DecValTok{1} \OperatorTok{|}\StringTok{ }\KeywordTok{is.na}\NormalTok{(POS_PCROUT)) }\OperatorTok{&}
\StringTok{                      }\NormalTok{(}\KeywordTok{is.na}\NormalTok{(PCR_VSR)) }\OperatorTok{&}
\StringTok{                      }\NormalTok{(}\KeywordTok{is.na}\NormalTok{(PCR_METAP)) }\OperatorTok{&}
\StringTok{                      }\NormalTok{(}\KeywordTok{is.na}\NormalTok{(PCR_PARA1))}
\NormalTok{                  )}
\NormalTok{  )}
\end{Highlighting}
\end{Shaded}

After this selection, 454830 cases are selected.

The next step consists of selecting female cases. The sex variable is
\texttt{CS\_SEXO}, in which F-Female, M-Male and I-Ignored.

\begin{Shaded}
\begin{Highlighting}[]
\NormalTok{questionr}\OperatorTok{::}\KeywordTok{freq}\NormalTok{(}
\NormalTok{  data3}\OperatorTok{$}\NormalTok{CS_SEXO,}
  \DataTypeTok{cum =} \OtherTok{FALSE}\NormalTok{,}
  \DataTypeTok{total =} \OtherTok{TRUE}\NormalTok{,}
  \DataTypeTok{na.last =} \OtherTok{FALSE}\NormalTok{,}
  \DataTypeTok{valid =} \OtherTok{FALSE}
\NormalTok{) }\OperatorTok
\StringTok{  }\NormalTok{knitr}\OperatorTok{::}\KeywordTok{kable}\NormalTok{(}\DataTypeTok{caption =} \StringTok{"Frequency table for sex"}\NormalTok{, }\DataTypeTok{digits =} \DecValTok{2}\NormalTok{) }\OperatorTok
\StringTok{  }\KeywordTok{kable_styling}\NormalTok{(}\DataTypeTok{latex_options =} \StringTok{"hold_position"}\NormalTok{)}
\end{Highlighting}
\end{Shaded}

\begin{table}[!h]

\caption{\label{tab:unnamed-chunk-6}Frequency table for sex}
\centering
\begin{tabular}[t]{l|r|r}
\hline
  & n & \%\\
\hline
F & 199931 & 44\\
\hline
I & 74 & 0\\
\hline
M & 254825 & 56\\
\hline
Total & 454830 & 100\\
\hline
\end{tabular}
\end{table}

\begin{Shaded}
\begin{Highlighting}[]
\CommentTok{#Filtering female cases}
\NormalTok{data4 <-}\StringTok{ }\NormalTok{dplyr}\OperatorTok{::}\KeywordTok{filter}\NormalTok{(data3, CS_SEXO }\OperatorTok{==}\StringTok{ "F"}\NormalTok{)}
\end{Highlighting}
\end{Shaded}

Now there are 199931 cases. The next selection is to consider female
people over 9 years old and under 50 (not inclusive). The variable that
indicates the cases age is \texttt{NU\_IDADE\_N}.

\begin{Shaded}
\begin{Highlighting}[]
\CommentTok{#Filtering female people over 9 years old and under 50}
\NormalTok{data5 <-}\StringTok{ }\NormalTok{dplyr}\OperatorTok{::}\KeywordTok{filter}\NormalTok{(data4, NU_IDADE_N }\OperatorTok{>}\StringTok{ }\DecValTok{9} \OperatorTok{&}\StringTok{ }\NormalTok{NU_IDADE_N }\OperatorTok{<}\StringTok{ }\DecValTok{50}\NormalTok{)}
\end{Highlighting}
\end{Shaded}

The number of cases results in 50845 cases.

Now we are going to identify pregnant people. For this, we will analyze
the variable \texttt{CS\_GESTANT}. This variable assumes the values:
1-1st gestational trimester; 2-2nd gestational trimester; 3-3rd
gestational trimester; 4-Ignored gestational age; 5-No; 6-Does not
apply; 9-Ignored.

\begin{Shaded}
\begin{Highlighting}[]
\NormalTok{questionr}\OperatorTok{::}\KeywordTok{freq}\NormalTok{(}
\NormalTok{  data5}\OperatorTok{$}\NormalTok{CS_GESTANT,}
  \DataTypeTok{cum =} \OtherTok{FALSE}\NormalTok{,}
  \DataTypeTok{total =} \OtherTok{TRUE}\NormalTok{,}
  \DataTypeTok{na.last =} \OtherTok{FALSE}\NormalTok{,}
  \DataTypeTok{valid =} \OtherTok{FALSE}
\NormalTok{) }\OperatorTok
\StringTok{  }\NormalTok{knitr}\OperatorTok{::}\KeywordTok{kable}\NormalTok{(}\DataTypeTok{caption =} \StringTok{"Frequency table for pregnancy variable"}\NormalTok{, }\DataTypeTok{digits =} \DecValTok{2}\NormalTok{) }\OperatorTok
\StringTok{  }\KeywordTok{kable_styling}\NormalTok{(}\DataTypeTok{latex_options =} \StringTok{"hold_position"}\NormalTok{)}
\end{Highlighting}
\end{Shaded}

\begin{table}[!h]

\caption{\label{tab:unnamed-chunk-9}Frequency table for pregnancy variable}
\centering
\begin{tabular}[t]{l|r|r}
\hline
  & n & \%\\
\hline
0 & 1 & 0.0\\
\hline
1 & 295 & 0.6\\
\hline
2 & 829 & 1.6\\
\hline
3 & 2089 & 4.1\\
\hline
4 & 159 & 0.3\\
\hline
5 & 37268 & 73.3\\
\hline
6 & 2056 & 4.0\\
\hline
9 & 8148 & 16.0\\
\hline
Total & 50845 & 100.0\\
\hline
\end{tabular}
\end{table}

The next step is filtering cases we have information about pregnancy
(yes - any gestational age - or not).

\begin{Shaded}
\begin{Highlighting}[]
\CommentTok{#Not considering do not apply and ignored}
\NormalTok{data6 <-}\StringTok{ }\NormalTok{dplyr}\OperatorTok{::}\KeywordTok{filter}\NormalTok{(data5, CS_GESTANT }\OperatorTok{>=}\StringTok{ }\DecValTok{1} \OperatorTok{&}\StringTok{ }\NormalTok{CS_GESTANT }\OperatorTok{<=}\StringTok{ }\DecValTok{5}\NormalTok{)}
\end{Highlighting}
\end{Shaded}

After the above filtering, we get 40640 observations.

The pregnancy indicator variable (independent of the gestational period)
is created below.

\begin{Shaded}
\begin{Highlighting}[]
\CommentTok{#Creating pregnancy indicator variable}
\NormalTok{data6 <-}\StringTok{ }\NormalTok{data6 }\OperatorTok
\StringTok{  }\NormalTok{dplyr}\OperatorTok{::}\KeywordTok{mutate}\NormalTok{(}\DataTypeTok{gestante_SN =} \KeywordTok{ifelse}\NormalTok{(CS_GESTANT }\OperatorTok{==}\StringTok{ }\DecValTok{5}\NormalTok{, }\StringTok{"no"}\NormalTok{, }\StringTok{"yes"}\NormalTok{))}
\end{Highlighting}
\end{Shaded}

\begin{Shaded}
\begin{Highlighting}[]
\NormalTok{questionr}\OperatorTok{::}\KeywordTok{freq}\NormalTok{(}
\NormalTok{  data6}\OperatorTok{$}\NormalTok{gestante_SN,}
  \DataTypeTok{cum =} \OtherTok{FALSE}\NormalTok{,}
  \DataTypeTok{total =} \OtherTok{TRUE}\NormalTok{,}
  \DataTypeTok{na.last =} \OtherTok{FALSE}\NormalTok{,}
  \DataTypeTok{valid =} \OtherTok{FALSE}
\NormalTok{) }\OperatorTok\StringTok{ }
\StringTok{  }\NormalTok{knitr}\OperatorTok{::}\KeywordTok{kable}\NormalTok{(}\DataTypeTok{caption =} \StringTok{"Frequency table for pregnancy indicator"}\NormalTok{, }\DataTypeTok{digits =} \DecValTok{2}\NormalTok{) }\OperatorTok
\StringTok{  }\KeywordTok{kable_styling}\NormalTok{(}\DataTypeTok{latex_options =} \StringTok{"hold_position"}\NormalTok{)}
\end{Highlighting}
\end{Shaded}

\begin{table}[!h]

\caption{\label{tab:unnamed-chunk-12}Frequency table for pregnancy indicator}
\centering
\begin{tabular}[t]{l|r|r}
\hline
  & n & \%\\
\hline
no & 37268 & 91.7\\
\hline
yes & 3372 & 8.3\\
\hline
Total & 40640 & 100.0\\
\hline
\end{tabular}
\end{table}

The next step is considering the postpartum indicator variable. The
\texttt{PUERPERA} variable has three categories: 1-yes, 2-no, and
9-Ignored.

\begin{Shaded}
\begin{Highlighting}[]
\NormalTok{questionr}\OperatorTok{::}\KeywordTok{freq}\NormalTok{(}
\NormalTok{  data6}\OperatorTok{$}\NormalTok{PUERPERA,}
  \DataTypeTok{cum =} \OtherTok{FALSE}\NormalTok{,}
  \DataTypeTok{total =} \OtherTok{TRUE}\NormalTok{,}
  \DataTypeTok{na.last =} \OtherTok{FALSE}\NormalTok{,}
  \DataTypeTok{valid =} \OtherTok{FALSE}
\NormalTok{) }\OperatorTok
\StringTok{  }\NormalTok{knitr}\OperatorTok{::}\KeywordTok{kable}\NormalTok{(}\DataTypeTok{caption =} \StringTok{"Frequency table for postpartum indicator"}\NormalTok{, }\DataTypeTok{digits =} \DecValTok{2}\NormalTok{) }\OperatorTok
\StringTok{  }\KeywordTok{kable_styling}\NormalTok{(}\DataTypeTok{latex_options =} \StringTok{"hold_position"}\NormalTok{)}
\end{Highlighting}
\end{Shaded}

\begin{table}[!h]

\caption{\label{tab:unnamed-chunk-13}Frequency table for postpartum indicator}
\centering
\begin{tabular}[t]{l|r|r}
\hline
  & n & \%\\
\hline
1 & 983 & 2.4\\
\hline
2 & 12183 & 30.0\\
\hline
9 & 230 & 0.6\\
\hline
NA & 27244 & 67.0\\
\hline
Total & 40640 & 100.0\\
\hline
\end{tabular}
\end{table}

Now we can create the group variable with the categories: preg - for
pregnant women, puerp - for postpartum, and no - for woman of
reprodutive age.

\begin{Shaded}
\begin{Highlighting}[]
\CommentTok{#Creating postpartum indicator}
\NormalTok{data6 <-}\StringTok{ }\NormalTok{data6 }\OperatorTok
\StringTok{  }\NormalTok{dplyr}\OperatorTok{::}\KeywordTok{mutate}\NormalTok{(}\DataTypeTok{puerpera =} \KeywordTok{ifelse}\NormalTok{(}\KeywordTok{is.na}\NormalTok{(PUERPERA) }\OperatorTok{==}\StringTok{ }\OtherTok{TRUE}\NormalTok{, }\DecValTok{0}\NormalTok{, PUERPERA))}

\CommentTok{#Creating group varible with three categories:}
\CommentTok{##preg - for pregnant women, }
\CommentTok{##puerp - for postpartum and }
\CommentTok{##no - woman of reprodutive age}
\NormalTok{data6 <-}\StringTok{ }\NormalTok{data6 }\OperatorTok
\StringTok{  }\NormalTok{dplyr}\OperatorTok{::}\KeywordTok{mutate}\NormalTok{(}\DataTypeTok{gest_puerp =} \KeywordTok{ifelse}\NormalTok{(}
\NormalTok{    gestante_SN }\OperatorTok{==}\StringTok{ "yes"}\NormalTok{,}
    \DecValTok{1}\NormalTok{,}
    \KeywordTok{ifelse}\NormalTok{(gestante_SN }\OperatorTok{==}\StringTok{ "no"} \OperatorTok{&}\StringTok{ }\NormalTok{puerpera }\OperatorTok{==}\StringTok{ }\DecValTok{1}\NormalTok{, }\DecValTok{2}\NormalTok{, }\DecValTok{0}\NormalTok{))}
\NormalTok{    )}
  
\NormalTok{data6}\OperatorTok{$}\NormalTok{gest_puerp <-}\StringTok{ }\KeywordTok{factor}\NormalTok{(data6}\OperatorTok{$}\NormalTok{gest_puerp, }
                     \DataTypeTok{levels =} \KeywordTok{c}\NormalTok{(}\DecValTok{0}\NormalTok{,}\DecValTok{1}\NormalTok{,}\DecValTok{2}\NormalTok{), }
                     \DataTypeTok{labels =} \KeywordTok{c}\NormalTok{(}\StringTok{"no"}\NormalTok{, }\StringTok{"preg"}\NormalTok{, }\StringTok{"puerp"}\NormalTok{))}
\end{Highlighting}
\end{Shaded}

\begin{Shaded}
\begin{Highlighting}[]
\NormalTok{questionr}\OperatorTok{::}\KeywordTok{freq}\NormalTok{(}
\NormalTok{  data6}\OperatorTok{$}\NormalTok{gest_puerp,}
  \DataTypeTok{cum =} \OtherTok{FALSE}\NormalTok{,}
  \DataTypeTok{total =} \OtherTok{TRUE}\NormalTok{,}
  \DataTypeTok{na.last =} \OtherTok{FALSE}\NormalTok{,}
  \DataTypeTok{valid =} \OtherTok{FALSE}
\NormalTok{) }\OperatorTok
\StringTok{  }\NormalTok{knitr}\OperatorTok{::}\KeywordTok{kable}\NormalTok{(}\DataTypeTok{caption =} \StringTok{"Frequency table for group variable"}\NormalTok{, }\DataTypeTok{digits =} \DecValTok{2}\NormalTok{) }\OperatorTok
\StringTok{  }\KeywordTok{kable_styling}\NormalTok{(}\DataTypeTok{latex_options =} \StringTok{"hold_position"}\NormalTok{)}
\end{Highlighting}
\end{Shaded}

\begin{table}[!h]

\caption{\label{tab:unnamed-chunk-15}Frequency table for group variable}
\centering
\begin{tabular}[t]{l|r|r}
\hline
  & n & \%\\
\hline
no & 36474 & 89.7\\
\hline
preg & 3372 & 8.3\\
\hline
puerp & 794 & 2.0\\
\hline
Total & 40640 & 100.0\\
\hline
\end{tabular}
\end{table}

\hypertarget{characterization-variables-and-comorbities}{%
\subsection{3.1 Characterization variables and
comorbities}\label{characterization-variables-and-comorbities}}

The age information is in \texttt{NU\_IDADE\_N}. We create the age group
variable (\texttt{faixa\_et}) with categories: ``\textless20'',
``20-34'' and ``\textgreater34''.

\begin{Shaded}
\begin{Highlighting}[]
\CommentTok{#age group variable}
\NormalTok{data6 <-}
\StringTok{  }\NormalTok{dplyr}\OperatorTok{::}\KeywordTok{mutate}\NormalTok{(data6, }\DataTypeTok{faixa_et =} \KeywordTok{ifelse}\NormalTok{(}
\NormalTok{    NU_IDADE_N }\OperatorTok{<=}\StringTok{ }\DecValTok{19}\NormalTok{,}
    \StringTok{"<20"}\NormalTok{,}
    \KeywordTok{ifelse}\NormalTok{(NU_IDADE_N }\OperatorTok{>=}\StringTok{ }\DecValTok{20} \OperatorTok{&}\StringTok{ }\NormalTok{NU_IDADE_N }\OperatorTok{<=}\StringTok{ }\DecValTok{34}\NormalTok{, }\StringTok{"20-34"}\NormalTok{, }\StringTok{">34"}\NormalTok{)}
\NormalTok{  ))}

\NormalTok{data6}\OperatorTok{$}\NormalTok{faixa_et <-}\StringTok{ }\KeywordTok{factor}\NormalTok{(data6}\OperatorTok{$}\NormalTok{faixa_et,}
                         \DataTypeTok{levels =} \KeywordTok{c}\NormalTok{(}\StringTok{"<20"}\NormalTok{, }\StringTok{"20-34"}\NormalTok{, }\StringTok{">34"}\NormalTok{))}
\end{Highlighting}
\end{Shaded}

\begin{Shaded}
\begin{Highlighting}[]
\NormalTok{questionr}\OperatorTok{::}\KeywordTok{freq}\NormalTok{(}
\NormalTok{  data6}\OperatorTok{$}\NormalTok{faixa_et,}
  \DataTypeTok{cum =} \OtherTok{FALSE}\NormalTok{,}
  \DataTypeTok{total =} \OtherTok{TRUE}\NormalTok{,}
  \DataTypeTok{na.last =} \OtherTok{FALSE}\NormalTok{,}
  \DataTypeTok{valid =} \OtherTok{FALSE}
\NormalTok{) }\OperatorTok
\StringTok{  }\NormalTok{knitr}\OperatorTok{::}\KeywordTok{kable}\NormalTok{(}\DataTypeTok{caption =} \StringTok{"Frequency table for group age"}\NormalTok{, }\DataTypeTok{digits =} \DecValTok{2}\NormalTok{) }\OperatorTok
\StringTok{  }\KeywordTok{kable_styling}\NormalTok{(}\DataTypeTok{latex_options =} \StringTok{"hold_position"}\NormalTok{)}
\end{Highlighting}
\end{Shaded}

\begin{table}[!h]

\caption{\label{tab:unnamed-chunk-17}Frequency table for group age}
\centering
\begin{tabular}[t]{l|r|r}
\hline
  & n & \%\\
\hline
<20 & 1320 & 3.2\\
\hline
20-34 & 12367 & 30.4\\
\hline
>34 & 26953 & 66.3\\
\hline
Total & 40640 & 100.0\\
\hline
\end{tabular}
\end{table}

For ethnicity (\texttt{CS\_RACA}), the categories are: 1-white; 2-black;
3-yellow; 4-brown; 5-Indigenous; 6-Ignored.

\begin{Shaded}
\begin{Highlighting}[]
\NormalTok{questionr}\OperatorTok{::}\KeywordTok{freq}\NormalTok{(}
\NormalTok{  data6}\OperatorTok{$}\NormalTok{CS_RACA,}
  \DataTypeTok{cum =} \OtherTok{FALSE}\NormalTok{,}
  \DataTypeTok{total =} \OtherTok{TRUE}\NormalTok{,}
  \DataTypeTok{na.last =} \OtherTok{FALSE}\NormalTok{,}
  \DataTypeTok{valid =} \OtherTok{FALSE}
\NormalTok{) }\OperatorTok
\StringTok{  }\NormalTok{knitr}\OperatorTok{::}\KeywordTok{kable}\NormalTok{(}\DataTypeTok{caption =} \StringTok{"Frequency table for ethnicity"}\NormalTok{, }\DataTypeTok{digits =} \DecValTok{2}\NormalTok{) }\OperatorTok
\StringTok{  }\KeywordTok{kable_styling}\NormalTok{(}\DataTypeTok{latex_options =} \StringTok{"hold_position"}\NormalTok{)}
\end{Highlighting}
\end{Shaded}

\begin{table}[!h]

\caption{\label{tab:unnamed-chunk-18}Frequency table for ethnicity}
\centering
\begin{tabular}[t]{l|r|r}
\hline
  & n & \%\\
\hline
1 & 16758 & 41.2\\
\hline
2 & 1917 & 4.7\\
\hline
3 & 396 & 1.0\\
\hline
4 & 12706 & 31.3\\
\hline
5 & 114 & 0.3\\
\hline
9 & 6135 & 15.1\\
\hline
NA & 2614 & 6.4\\
\hline
Total & 40640 & 100.0\\
\hline
\end{tabular}
\end{table}

We will now label this variable, creating the variable \texttt{raca},
considering only the valid categories.

\begin{Shaded}
\begin{Highlighting}[]
\CommentTok{#ethnicity variable}
\NormalTok{data6}\OperatorTok{$}\NormalTok{raca <-}\StringTok{ }\KeywordTok{factor}\NormalTok{(}
\NormalTok{  data6}\OperatorTok{$}\NormalTok{CS_RACA,}
  \DataTypeTok{levels =} \KeywordTok{c}\NormalTok{(}\StringTok{"1"}\NormalTok{, }\StringTok{"2"}\NormalTok{, }\StringTok{"3"}\NormalTok{, }\StringTok{"4"}\NormalTok{, }\StringTok{"5"}\NormalTok{),}
  \DataTypeTok{labels =} \KeywordTok{c}\NormalTok{(}\StringTok{"white"}\NormalTok{, }\StringTok{"black"}\NormalTok{, }\StringTok{"yellow"}\NormalTok{, }\StringTok{"brown"}\NormalTok{, }\StringTok{"indigenous"}\NormalTok{)}
\NormalTok{)}
\end{Highlighting}
\end{Shaded}

For education (\texttt{CS\_ESCOL\_N}), the categories are: 0-no
education/illiterate; 1-fundamental 1st cycle; 2-fundamental 2nd cycle;
3-high school; 4-superior; 5-not applicable, 9-ignored.

\begin{Shaded}
\begin{Highlighting}[]
\NormalTok{questionr}\OperatorTok{::}\KeywordTok{freq}\NormalTok{(}
\NormalTok{  data6}\OperatorTok{$}\NormalTok{CS_ESCOL_N,}
  \DataTypeTok{cum =} \OtherTok{FALSE}\NormalTok{,}
  \DataTypeTok{total =} \OtherTok{TRUE}\NormalTok{,}
  \DataTypeTok{na.last =} \OtherTok{FALSE}\NormalTok{,}
  \DataTypeTok{valid =} \OtherTok{FALSE}
\NormalTok{) }\OperatorTok
\StringTok{  }\NormalTok{knitr}\OperatorTok{::}\KeywordTok{kable}\NormalTok{(}\DataTypeTok{caption =} \StringTok{"Frequency table for school"}\NormalTok{, }\DataTypeTok{digits =} \DecValTok{2}\NormalTok{) }\OperatorTok
\StringTok{  }\KeywordTok{kable_styling}\NormalTok{(}\DataTypeTok{latex_options =} \StringTok{"hold_position"}\NormalTok{)}
\end{Highlighting}
\end{Shaded}

\begin{table}[!h]

\caption{\label{tab:unnamed-chunk-20}Frequency table for school}
\centering
\begin{tabular}[t]{l|r|r}
\hline
  & n & \%\\
\hline
0 & 228 & 0.6\\
\hline
1 & 1546 & 3.8\\
\hline
2 & 2556 & 6.3\\
\hline
3 & 8501 & 20.9\\
\hline
4 & 4403 & 10.8\\
\hline
5 & 1 & 0.0\\
\hline
9 & 11495 & 28.3\\
\hline
NA & 11910 & 29.3\\
\hline
Total & 40640 & 100.0\\
\hline
\end{tabular}
\end{table}

We will now label this variable, creating the variable \texttt{escol},
considering only the valid categories and considering the following
categories: no education/illiterate (\texttt{CS\_ESCOL\_N} = 0), up to
high school (\texttt{CS\_ESCOL\_N} = 1 or 2), high school
(\texttt{CS\_ESCOL\_N} = 3) and higher education (\texttt{CS\_ESCOL\_N}
= 4).

\begin{Shaded}
\begin{Highlighting}[]
\CommentTok{#school variable}
\NormalTok{data6}\OperatorTok{$}\NormalTok{escol <-}\StringTok{ }\KeywordTok{factor}\NormalTok{(}
\NormalTok{  data6}\OperatorTok{$}\NormalTok{CS_ESCOL_N,}
  \DataTypeTok{levels =} \KeywordTok{c}\NormalTok{(}\StringTok{"0"}\NormalTok{, }\StringTok{"1"}\NormalTok{, }\StringTok{"2"}\NormalTok{, }\StringTok{"3"}\NormalTok{, }\StringTok{"4"}\NormalTok{),}
  \DataTypeTok{labels =} \KeywordTok{c}\NormalTok{(}
    \StringTok{"no education"}\NormalTok{,}
    \StringTok{"up to high school"}\NormalTok{,}
    \StringTok{"up to high school"}\NormalTok{,}
    \StringTok{"high school"}\NormalTok{,}
    \StringTok{"higher education"}
\NormalTok{  )}
\NormalTok{)}
\end{Highlighting}
\end{Shaded}

\begin{Shaded}
\begin{Highlighting}[]
\NormalTok{questionr}\OperatorTok{::}\KeywordTok{freq}\NormalTok{(}
\NormalTok{  data6}\OperatorTok{$}\NormalTok{escol,}
  \DataTypeTok{cum =} \OtherTok{FALSE}\NormalTok{,}
  \DataTypeTok{total =} \OtherTok{TRUE}\NormalTok{,}
  \DataTypeTok{na.last =} \OtherTok{FALSE}\NormalTok{,}
  \DataTypeTok{valid =} \OtherTok{FALSE}
\NormalTok{) }\OperatorTok
\StringTok{  }\NormalTok{knitr}\OperatorTok{::}\KeywordTok{kable}\NormalTok{(}\DataTypeTok{caption =} \StringTok{"Frequency table for school (new categories)"}\NormalTok{, }\DataTypeTok{digits =} \DecValTok{2}\NormalTok{) }\OperatorTok
\StringTok{  }\KeywordTok{kable_styling}\NormalTok{(}\DataTypeTok{latex_options =} \StringTok{"hold_position"}\NormalTok{)}
\end{Highlighting}
\end{Shaded}

\begin{table}[!h]

\caption{\label{tab:unnamed-chunk-22}Frequency table for school (new categories)}
\centering
\begin{tabular}[t]{l|r|r}
\hline
  & n & \%\\
\hline
no education & 228 & 0.6\\
\hline
up to high school & 4102 & 10.1\\
\hline
high school & 8501 & 20.9\\
\hline
higher education & 4403 & 10.8\\
\hline
NA & 23406 & 57.6\\
\hline
Total & 40640 & 100.0\\
\hline
\end{tabular}
\end{table}

For comorbidities, the categories are: 1-yes, 2-no and 9-ignored. The
comorbidities considered are: cardiopathy, hematology, liver disease,
asthma, diabetes, neurological diseases, pneumopathy, immunosuppression,
kidney disease, and obesity, and their frequency tables are presented
below, respectively:

\begin{Shaded}
\begin{Highlighting}[]
\NormalTok{questionr}\OperatorTok{::}\KeywordTok{freq}\NormalTok{(}
\NormalTok{  data6}\OperatorTok{$}\NormalTok{CARDIOPATI,}
  \DataTypeTok{cum =} \OtherTok{FALSE}\NormalTok{,}
  \DataTypeTok{total =} \OtherTok{TRUE}\NormalTok{,}
  \DataTypeTok{na.last =} \OtherTok{FALSE}\NormalTok{,}
  \DataTypeTok{valid =} \OtherTok{FALSE}
\NormalTok{) }\OperatorTok
\StringTok{  }\NormalTok{knitr}\OperatorTok{::}\KeywordTok{kable}\NormalTok{(}\DataTypeTok{caption =} \StringTok{"Frequency table for cardiopathy"}\NormalTok{, }\DataTypeTok{digits =} \DecValTok{2}\NormalTok{) }\OperatorTok
\StringTok{  }\KeywordTok{kable_styling}\NormalTok{(}\DataTypeTok{latex_options =} \StringTok{"hold_position"}\NormalTok{)}
\end{Highlighting}
\end{Shaded}

\begin{table}[!h]

\caption{\label{tab:unnamed-chunk-23}Frequency table for cardiopathy}
\centering
\begin{tabular}[t]{l|r|r}
\hline
  & n & \%\\
\hline
1 & 6071 & 14.9\\
\hline
2 & 8905 & 21.9\\
\hline
9 & 230 & 0.6\\
\hline
NA & 25434 & 62.6\\
\hline
Total & 40640 & 100.0\\
\hline
\end{tabular}
\end{table}

\begin{Shaded}
\begin{Highlighting}[]
\NormalTok{questionr}\OperatorTok{::}\KeywordTok{freq}\NormalTok{(}
\NormalTok{  data6}\OperatorTok{$}\NormalTok{HEMATOLOGI,}
  \DataTypeTok{cum =} \OtherTok{FALSE}\NormalTok{,}
  \DataTypeTok{total =} \OtherTok{TRUE}\NormalTok{,}
  \DataTypeTok{na.last =} \OtherTok{FALSE}\NormalTok{,}
  \DataTypeTok{valid =} \OtherTok{FALSE}
\NormalTok{) }\OperatorTok
\StringTok{  }\NormalTok{knitr}\OperatorTok{::}\KeywordTok{kable}\NormalTok{(}\DataTypeTok{caption =} \StringTok{"Frequency table for hematology"}\NormalTok{, }\DataTypeTok{digits =} \DecValTok{2}\NormalTok{) }\OperatorTok
\StringTok{  }\KeywordTok{kable_styling}\NormalTok{(}\DataTypeTok{latex_options =} \StringTok{"hold_position"}\NormalTok{)}
\end{Highlighting}
\end{Shaded}

\begin{table}[!h]

\caption{\label{tab:unnamed-chunk-24}Frequency table for hematology}
\centering
\begin{tabular}[t]{l|r|r}
\hline
  & n & \%\\
\hline
1 & 383 & 0.9\\
\hline
2 & 12518 & 30.8\\
\hline
9 & 282 & 0.7\\
\hline
NA & 27457 & 67.6\\
\hline
Total & 40640 & 100.0\\
\hline
\end{tabular}
\end{table}

\begin{Shaded}
\begin{Highlighting}[]
\NormalTok{questionr}\OperatorTok{::}\KeywordTok{freq}\NormalTok{(}
\NormalTok{  data6}\OperatorTok{$}\NormalTok{HEPATICA,}
  \DataTypeTok{cum =} \OtherTok{FALSE}\NormalTok{,}
  \DataTypeTok{total =} \OtherTok{TRUE}\NormalTok{,}
  \DataTypeTok{na.last =} \OtherTok{FALSE}\NormalTok{,}
  \DataTypeTok{valid =} \OtherTok{FALSE}
\NormalTok{) }\OperatorTok
\StringTok{  }\NormalTok{knitr}\OperatorTok{::}\KeywordTok{kable}\NormalTok{(}\DataTypeTok{caption =} \StringTok{"Frequency table for liver disease"}\NormalTok{, }\DataTypeTok{digits =} \DecValTok{2}\NormalTok{) }\OperatorTok
\StringTok{  }\KeywordTok{kable_styling}\NormalTok{(}\DataTypeTok{latex_options =} \StringTok{"hold_position"}\NormalTok{)}
\end{Highlighting}
\end{Shaded}

\begin{table}[!h]

\caption{\label{tab:unnamed-chunk-25}Frequency table for liver disease}
\centering
\begin{tabular}[t]{l|r|r}
\hline
  & n & \%\\
\hline
1 & 201 & 0.5\\
\hline
2 & 12568 & 30.9\\
\hline
9 & 286 & 0.7\\
\hline
NA & 27585 & 67.9\\
\hline
Total & 40640 & 100.0\\
\hline
\end{tabular}
\end{table}

\begin{Shaded}
\begin{Highlighting}[]
\NormalTok{questionr}\OperatorTok{::}\KeywordTok{freq}\NormalTok{(}
\NormalTok{  data6}\OperatorTok{$}\NormalTok{ASMA,}
  \DataTypeTok{cum =} \OtherTok{FALSE}\NormalTok{,}
  \DataTypeTok{total =} \OtherTok{TRUE}\NormalTok{,}
  \DataTypeTok{na.last =} \OtherTok{FALSE}\NormalTok{,}
  \DataTypeTok{valid =} \OtherTok{FALSE}
\NormalTok{) }\OperatorTok
\StringTok{  }\NormalTok{knitr}\OperatorTok{::}\KeywordTok{kable}\NormalTok{(}\DataTypeTok{caption =} \StringTok{"Frequency table for asthma"}\NormalTok{, }\DataTypeTok{digits =} \DecValTok{2}\NormalTok{) }\OperatorTok
\StringTok{  }\KeywordTok{kable_styling}\NormalTok{(}\DataTypeTok{latex_options =} \StringTok{"hold_position"}\NormalTok{)}
\end{Highlighting}
\end{Shaded}

\begin{table}[!h]

\caption{\label{tab:unnamed-chunk-26}Frequency table for asthma}
\centering
\begin{tabular}[t]{l|r|r}
\hline
  & n & \%\\
\hline
1 & 2055 & 5.1\\
\hline
2 & 11371 & 28.0\\
\hline
9 & 264 & 0.6\\
\hline
NA & 26950 & 66.3\\
\hline
Total & 40640 & 100.0\\
\hline
\end{tabular}
\end{table}

\begin{Shaded}
\begin{Highlighting}[]
\NormalTok{questionr}\OperatorTok{::}\KeywordTok{freq}\NormalTok{(}
\NormalTok{  data6}\OperatorTok{$}\NormalTok{DIABETES,}
  \DataTypeTok{cum =} \OtherTok{FALSE}\NormalTok{,}
  \DataTypeTok{total =} \OtherTok{TRUE}\NormalTok{,}
  \DataTypeTok{na.last =} \OtherTok{FALSE}\NormalTok{,}
  \DataTypeTok{valid =} \OtherTok{FALSE}
\NormalTok{) }\OperatorTok
\StringTok{  }\NormalTok{knitr}\OperatorTok{::}\KeywordTok{kable}\NormalTok{(}\DataTypeTok{caption =} \StringTok{"Frequency table for diabetes"}\NormalTok{, }\DataTypeTok{digits =} \DecValTok{2}\NormalTok{) }\OperatorTok
\StringTok{  }\KeywordTok{kable_styling}\NormalTok{(}\DataTypeTok{latex_options =} \StringTok{"hold_position"}\NormalTok{)}
\end{Highlighting}
\end{Shaded}

\begin{table}[!h]

\caption{\label{tab:unnamed-chunk-27}Frequency table for diabetes}
\centering
\begin{tabular}[t]{l|r|r}
\hline
  & n & \%\\
\hline
1 & 5127 & 12.6\\
\hline
2 & 9448 & 23.2\\
\hline
9 & 217 & 0.5\\
\hline
NA & 25848 & 63.6\\
\hline
Total & 40640 & 100.0\\
\hline
\end{tabular}
\end{table}

\begin{Shaded}
\begin{Highlighting}[]
\NormalTok{questionr}\OperatorTok{::}\KeywordTok{freq}\NormalTok{(}
\NormalTok{  data6}\OperatorTok{$}\NormalTok{NEUROLOGIC,}
  \DataTypeTok{cum =} \OtherTok{FALSE}\NormalTok{,}
  \DataTypeTok{total =} \OtherTok{TRUE}\NormalTok{,}
  \DataTypeTok{na.last =} \OtherTok{FALSE}\NormalTok{,}
  \DataTypeTok{valid =} \OtherTok{FALSE}
\NormalTok{) }\OperatorTok
\StringTok{  }\NormalTok{knitr}\OperatorTok{::}\KeywordTok{kable}\NormalTok{(}\DataTypeTok{caption =} \StringTok{"Frequency table for neurological diseases"}\NormalTok{, }\DataTypeTok{digits =} \DecValTok{2}\NormalTok{) }\OperatorTok
\StringTok{  }\KeywordTok{kable_styling}\NormalTok{(}\DataTypeTok{latex_options =} \StringTok{"hold_position"}\NormalTok{)}
\end{Highlighting}
\end{Shaded}

\begin{table}[!h]

\caption{\label{tab:unnamed-chunk-28}Frequency table for neurological diseases}
\centering
\begin{tabular}[t]{l|r|r}
\hline
  & n & \%\\
\hline
1 & 598 & 1.5\\
\hline
2 & 12335 & 30.4\\
\hline
9 & 273 & 0.7\\
\hline
NA & 27434 & 67.5\\
\hline
Total & 40640 & 100.0\\
\hline
\end{tabular}
\end{table}

\begin{Shaded}
\begin{Highlighting}[]
\NormalTok{questionr}\OperatorTok{::}\KeywordTok{freq}\NormalTok{(}
\NormalTok{  data6}\OperatorTok{$}\NormalTok{PNEUMOPATI,}
  \DataTypeTok{cum =} \OtherTok{FALSE}\NormalTok{,}
  \DataTypeTok{total =} \OtherTok{TRUE}\NormalTok{,}
  \DataTypeTok{na.last =} \OtherTok{FALSE}\NormalTok{,}
  \DataTypeTok{valid =} \OtherTok{FALSE}
\NormalTok{) }\OperatorTok
\StringTok{  }\NormalTok{knitr}\OperatorTok{::}\KeywordTok{kable}\NormalTok{(}\DataTypeTok{caption =} \StringTok{"Frequency table for pneumopathy"}\NormalTok{, }\DataTypeTok{digits =} \DecValTok{2}\NormalTok{) }\OperatorTok
\StringTok{  }\KeywordTok{kable_styling}\NormalTok{(}\DataTypeTok{latex_options =} \StringTok{"hold_position"}\NormalTok{)}
\end{Highlighting}
\end{Shaded}

\begin{table}[!h]

\caption{\label{tab:unnamed-chunk-29}Frequency table for pneumopathy}
\centering
\begin{tabular}[t]{l|r|r}
\hline
  & n & \%\\
\hline
1 & 607 & 1.5\\
\hline
2 & 12338 & 30.4\\
\hline
9 & 281 & 0.7\\
\hline
NA & 27414 & 67.5\\
\hline
Total & 40640 & 100.0\\
\hline
\end{tabular}
\end{table}

\begin{Shaded}
\begin{Highlighting}[]
\NormalTok{questionr}\OperatorTok{::}\KeywordTok{freq}\NormalTok{(}
\NormalTok{  data6}\OperatorTok{$}\NormalTok{IMUNODEPRE,}
  \DataTypeTok{cum =} \OtherTok{FALSE}\NormalTok{,}
  \DataTypeTok{total =} \OtherTok{TRUE}\NormalTok{,}
  \DataTypeTok{na.last =} \OtherTok{FALSE}\NormalTok{,}
  \DataTypeTok{valid =} \OtherTok{FALSE}
\NormalTok{) }\OperatorTok
\StringTok{  }\NormalTok{knitr}\OperatorTok{::}\KeywordTok{kable}\NormalTok{(}\DataTypeTok{caption =} \StringTok{"Frequency table for immunosuppression"}\NormalTok{, }\DataTypeTok{digits =} \DecValTok{2}\NormalTok{) }\OperatorTok
\StringTok{  }\KeywordTok{kable_styling}\NormalTok{(}\DataTypeTok{latex_options =} \StringTok{"hold_position"}\NormalTok{)}
\end{Highlighting}
\end{Shaded}

\begin{table}[!h]

\caption{\label{tab:unnamed-chunk-30}Frequency table for immunosuppression}
\centering
\begin{tabular}[t]{l|r|r}
\hline
  & n & \%\\
\hline
1 & 1347 & 3.3\\
\hline
2 & 11790 & 29.0\\
\hline
9 & 281 & 0.7\\
\hline
NA & 27222 & 67.0\\
\hline
Total & 40640 & 100.0\\
\hline
\end{tabular}
\end{table}

\begin{Shaded}
\begin{Highlighting}[]
\NormalTok{questionr}\OperatorTok{::}\KeywordTok{freq}\NormalTok{(}
\NormalTok{  data6}\OperatorTok{$}\NormalTok{RENAL,}
  \DataTypeTok{cum =} \OtherTok{FALSE}\NormalTok{,}
  \DataTypeTok{total =} \OtherTok{TRUE}\NormalTok{,}
  \DataTypeTok{na.last =} \OtherTok{FALSE}\NormalTok{,}
  \DataTypeTok{valid =} \OtherTok{FALSE}
\NormalTok{) }\OperatorTok
\StringTok{  }\NormalTok{knitr}\OperatorTok{::}\KeywordTok{kable}\NormalTok{(}\DataTypeTok{caption =} \StringTok{"Frequency table for kidney disease"}\NormalTok{, }\DataTypeTok{digits =} \DecValTok{2}\NormalTok{) }\OperatorTok
\StringTok{  }\KeywordTok{kable_styling}\NormalTok{(}\DataTypeTok{latex_options =} \StringTok{"hold_position"}\NormalTok{)}
\end{Highlighting}
\end{Shaded}

\begin{table}[!h]

\caption{\label{tab:unnamed-chunk-31}Frequency table for kidney disease}
\centering
\begin{tabular}[t]{l|r|r}
\hline
  & n & \%\\
\hline
1 & 1116 & 2.7\\
\hline
2 & 11931 & 29.4\\
\hline
9 & 275 & 0.7\\
\hline
NA & 27318 & 67.2\\
\hline
Total & 40640 & 100.0\\
\hline
\end{tabular}
\end{table}

\begin{Shaded}
\begin{Highlighting}[]
\NormalTok{questionr}\OperatorTok{::}\KeywordTok{freq}\NormalTok{(}
\NormalTok{  data6}\OperatorTok{$}\NormalTok{OBESIDADE,}
  \DataTypeTok{cum =} \OtherTok{FALSE}\NormalTok{,}
  \DataTypeTok{total =} \OtherTok{TRUE}\NormalTok{,}
  \DataTypeTok{na.last =} \OtherTok{FALSE}\NormalTok{,}
  \DataTypeTok{valid =} \OtherTok{FALSE}
\NormalTok{) }\OperatorTok
\StringTok{  }\NormalTok{knitr}\OperatorTok{::}\KeywordTok{kable}\NormalTok{(}\DataTypeTok{caption =} \StringTok{"Frequency table for obesity"}\NormalTok{, }\DataTypeTok{digits =} \DecValTok{2}\NormalTok{) }\OperatorTok
\StringTok{  }\KeywordTok{kable_styling}\NormalTok{(}\DataTypeTok{latex_options =} \StringTok{"hold_position"}\NormalTok{)}
\end{Highlighting}
\end{Shaded}

\begin{table}[!h]

\caption{\label{tab:unnamed-chunk-32}Frequency table for obesity}
\centering
\begin{tabular}[t]{l|r|r}
\hline
  & n & \%\\
\hline
1 & 3937 & 9.7\\
\hline
2 & 9776 & 24.1\\
\hline
9 & 462 & 1.1\\
\hline
NA & 26465 & 65.1\\
\hline
Total & 40640 & 100.0\\
\hline
\end{tabular}
\end{table}

We will label in the following the comorbidities indicators, considering
only the valid categories.

\begin{Shaded}
\begin{Highlighting}[]
\CommentTok{#cardiopathy}
\NormalTok{data6}\OperatorTok{$}\NormalTok{cardiopati <-}\StringTok{ }\KeywordTok{factor}\NormalTok{(data6}\OperatorTok{$}\NormalTok{CARDIOPATI,}
                           \DataTypeTok{levels =} \KeywordTok{c}\NormalTok{(}\StringTok{"1"}\NormalTok{, }\StringTok{"2"}\NormalTok{),}
                           \DataTypeTok{labels =} \KeywordTok{c}\NormalTok{(}\StringTok{"yes"}\NormalTok{, }\StringTok{"no"}\NormalTok{))}
\end{Highlighting}
\end{Shaded}

\begin{Shaded}
\begin{Highlighting}[]
\CommentTok{#hematology}
\NormalTok{data6}\OperatorTok{$}\NormalTok{hematologi <-}\StringTok{ }\KeywordTok{factor}\NormalTok{(data6}\OperatorTok{$}\NormalTok{HEMATOLOGI,}
                           \DataTypeTok{levels =} \KeywordTok{c}\NormalTok{(}\StringTok{"1"}\NormalTok{, }\StringTok{"2"}\NormalTok{),}
                           \DataTypeTok{labels =} \KeywordTok{c}\NormalTok{(}\StringTok{"yes"}\NormalTok{, }\StringTok{"no"}\NormalTok{))}
\end{Highlighting}
\end{Shaded}

\begin{Shaded}
\begin{Highlighting}[]
\CommentTok{#liver disease}
\NormalTok{data6}\OperatorTok{$}\NormalTok{hepatica <-}\StringTok{ }\KeywordTok{factor}\NormalTok{(data6}\OperatorTok{$}\NormalTok{HEPATICA,}
                         \DataTypeTok{levels =} \KeywordTok{c}\NormalTok{(}\StringTok{"1"}\NormalTok{, }\StringTok{"2"}\NormalTok{),}
                         \DataTypeTok{labels =} \KeywordTok{c}\NormalTok{(}\StringTok{"yes"}\NormalTok{, }\StringTok{"no"}\NormalTok{))}
\end{Highlighting}
\end{Shaded}

\begin{Shaded}
\begin{Highlighting}[]
\CommentTok{#asthma}
\NormalTok{data6}\OperatorTok{$}\NormalTok{asma <-}\StringTok{ }\KeywordTok{factor}\NormalTok{(data6}\OperatorTok{$}\NormalTok{ASMA,}
                     \DataTypeTok{levels =} \KeywordTok{c}\NormalTok{(}\StringTok{"1"}\NormalTok{, }\StringTok{"2"}\NormalTok{),}
                     \DataTypeTok{labels =} \KeywordTok{c}\NormalTok{(}\StringTok{"yes"}\NormalTok{, }\StringTok{"no"}\NormalTok{))}
\end{Highlighting}
\end{Shaded}

\begin{Shaded}
\begin{Highlighting}[]
\CommentTok{#diabetes}
\NormalTok{data6}\OperatorTok{$}\NormalTok{diabetes <-}\StringTok{ }\KeywordTok{factor}\NormalTok{(data6}\OperatorTok{$}\NormalTok{DIABETES,}
                         \DataTypeTok{levels =} \KeywordTok{c}\NormalTok{(}\StringTok{"1"}\NormalTok{, }\StringTok{"2"}\NormalTok{),}
                         \DataTypeTok{labels =} \KeywordTok{c}\NormalTok{(}\StringTok{"yes"}\NormalTok{, }\StringTok{"no"}\NormalTok{))}
\end{Highlighting}
\end{Shaded}

\begin{Shaded}
\begin{Highlighting}[]
\CommentTok{#neurological diseases}
\NormalTok{data6}\OperatorTok{$}\NormalTok{neuro <-}\StringTok{ }\KeywordTok{factor}\NormalTok{(data6}\OperatorTok{$}\NormalTok{NEUROLOGIC,}
                      \DataTypeTok{levels =} \KeywordTok{c}\NormalTok{(}\StringTok{"1"}\NormalTok{, }\StringTok{"2"}\NormalTok{),}
                      \DataTypeTok{labels =} \KeywordTok{c}\NormalTok{(}\StringTok{"yes"}\NormalTok{, }\StringTok{"no"}\NormalTok{))}
\end{Highlighting}
\end{Shaded}

\begin{Shaded}
\begin{Highlighting}[]
\CommentTok{#pneumopathy}
\NormalTok{data6}\OperatorTok{$}\NormalTok{pneumopati <-}\StringTok{ }\KeywordTok{factor}\NormalTok{(data6}\OperatorTok{$}\NormalTok{PNEUMOPATI,}
                           \DataTypeTok{levels =} \KeywordTok{c}\NormalTok{(}\StringTok{"1"}\NormalTok{, }\StringTok{"2"}\NormalTok{),}
                           \DataTypeTok{labels =} \KeywordTok{c}\NormalTok{(}\StringTok{"yes"}\NormalTok{, }\StringTok{"no"}\NormalTok{))}
\end{Highlighting}
\end{Shaded}

\begin{Shaded}
\begin{Highlighting}[]
\CommentTok{#immunosuppression}
\NormalTok{data6}\OperatorTok{$}\NormalTok{imunodepre <-}\StringTok{ }\KeywordTok{factor}\NormalTok{(data6}\OperatorTok{$}\NormalTok{IMUNODEPRE,}
                           \DataTypeTok{levels =} \KeywordTok{c}\NormalTok{(}\StringTok{"1"}\NormalTok{, }\StringTok{"2"}\NormalTok{),}
                           \DataTypeTok{labels =} \KeywordTok{c}\NormalTok{(}\StringTok{"yes"}\NormalTok{, }\StringTok{"no"}\NormalTok{))}
\end{Highlighting}
\end{Shaded}

\begin{Shaded}
\begin{Highlighting}[]
\CommentTok{#kidney disease}
\NormalTok{data6}\OperatorTok{$}\NormalTok{renal <-}\StringTok{ }\KeywordTok{factor}\NormalTok{(data6}\OperatorTok{$}\NormalTok{RENAL,}
                      \DataTypeTok{levels =} \KeywordTok{c}\NormalTok{(}\StringTok{"1"}\NormalTok{, }\StringTok{"2"}\NormalTok{),}
                      \DataTypeTok{labels =} \KeywordTok{c}\NormalTok{(}\StringTok{"yes"}\NormalTok{, }\StringTok{"no"}\NormalTok{))}
\end{Highlighting}
\end{Shaded}

\begin{Shaded}
\begin{Highlighting}[]
\CommentTok{#obesity}
\NormalTok{data6}\OperatorTok{$}\NormalTok{obesidade <-}\StringTok{ }\KeywordTok{factor}\NormalTok{(data6}\OperatorTok{$}\NormalTok{OBESIDADE,}
                          \DataTypeTok{levels =} \KeywordTok{c}\NormalTok{(}\StringTok{"1"}\NormalTok{, }\StringTok{"2"}\NormalTok{),}
                          \DataTypeTok{labels =} \KeywordTok{c}\NormalTok{(}\StringTok{"yes"}\NormalTok{, }\StringTok{"no"}\NormalTok{))}
\end{Highlighting}
\end{Shaded}

One variable we want to analyze is the comorbities group
(\texttt{gr\_comorb}) with the categories: ``none'', ``1 or 2'',
``\textgreater2''.

\begin{Shaded}
\begin{Highlighting}[]
\NormalTok{comorbidades <-}
\StringTok{  }\KeywordTok{c}\NormalTok{(}
    \StringTok{"CARDIOPATI_aux"}\NormalTok{,}
    \StringTok{"HEMATOLOGI_aux"}\NormalTok{,}
    \StringTok{"HEPATICA_aux"}\NormalTok{,}
    \StringTok{"ASMA_aux"}\NormalTok{,}
    \StringTok{"DIABETES_aux"}\NormalTok{,}
    \StringTok{"NEUROLOGIC_aux"}\NormalTok{,}
    \StringTok{"PNEUMOPATI_aux"}\NormalTok{,}
    \StringTok{"IMUNODEPRE_aux"}\NormalTok{,}
    \StringTok{"RENAL_aux"}\NormalTok{,}
    \StringTok{"OBESIDADE_aux"}
\NormalTok{  )}

\NormalTok{comorbidades1 <-}
\StringTok{  }\KeywordTok{c}\NormalTok{(}
    \StringTok{"CARDIOPATI_aux1"}\NormalTok{,}
    \StringTok{"HEMATOLOGI_aux1"}\NormalTok{,}
    \StringTok{"HEPATICA_aux1"}\NormalTok{,}
    \StringTok{"ASMA_aux1"}\NormalTok{,}
    \StringTok{"DIABETES_aux1"}\NormalTok{,}
    \StringTok{"NEUROLOGIC_aux1"}\NormalTok{,}
    \StringTok{"PNEUMOPATI_aux1"}\NormalTok{,}
    \StringTok{"IMUNODEPRE_aux1"}\NormalTok{,}
    \StringTok{"RENAL_aux1"}\NormalTok{,}
    \StringTok{"OBESIDADE_aux1"}
\NormalTok{  )}

\NormalTok{data6 <-}
\StringTok{  }\KeywordTok{mutate}\NormalTok{(}
\NormalTok{    data6,}
    \DataTypeTok{CARDIOPATI_aux =}\NormalTok{ CARDIOPATI,}
    \DataTypeTok{HEMATOLOGI_aux =}\NormalTok{ HEMATOLOGI,}
    \DataTypeTok{HEPATICA_aux =}\NormalTok{ HEPATICA,}
    \DataTypeTok{ASMA_aux =}\NormalTok{ ASMA,}
    \DataTypeTok{DIABETES_aux =}\NormalTok{ DIABETES,}
    \DataTypeTok{NEUROLOGIC_aux =}\NormalTok{ NEUROLOGIC,}
    \DataTypeTok{PNEUMOPATI_aux =}\NormalTok{ PNEUMOPATI,}
    \DataTypeTok{IMUNODEPRE_aux =}\NormalTok{ IMUNODEPRE,}
    \DataTypeTok{RENAL_aux =}\NormalTok{ RENAL,}
    \DataTypeTok{OBESIDADE_aux =}\NormalTok{ OBESIDADE}
\NormalTok{  )}

\NormalTok{data6 <-}
\StringTok{  }\KeywordTok{mutate}\NormalTok{(}
\NormalTok{    data6,}
    \DataTypeTok{CARDIOPATI_aux1 =}\NormalTok{ CARDIOPATI,}
    \DataTypeTok{HEMATOLOGI_aux1 =}\NormalTok{ HEMATOLOGI,}
    \DataTypeTok{HEPATICA_aux1 =}\NormalTok{ HEPATICA,}
    \DataTypeTok{ASMA_aux1 =}\NormalTok{ ASMA,}
    \DataTypeTok{DIABETES_aux1 =}\NormalTok{ DIABETES,}
    \DataTypeTok{NEUROLOGIC_aux1 =}\NormalTok{ NEUROLOGIC,}
    \DataTypeTok{PNEUMOPATI_aux1 =}\NormalTok{ PNEUMOPATI,}
    \DataTypeTok{IMUNODEPRE_aux1 =}\NormalTok{ IMUNODEPRE,}
    \DataTypeTok{RENAL_aux1 =}\NormalTok{ RENAL,}
    \DataTypeTok{OBESIDADE_aux1 =}\NormalTok{ OBESIDADE}
\NormalTok{  )}

\NormalTok{data6 <-}\StringTok{  }\NormalTok{data6 }\OperatorTok
\StringTok{  }\NormalTok{dplyr}\OperatorTok{::}\KeywordTok{mutate_at}\NormalTok{(dplyr}\OperatorTok{::}\KeywordTok{all_of}\NormalTok{(comorbidades), }\ControlFlowTok{function}\NormalTok{(x) \{}
\NormalTok{    dplyr}\OperatorTok{::}\KeywordTok{case_when}\NormalTok{(x }\OperatorTok{==}\StringTok{ "1"} \OperatorTok{~}\StringTok{ }\DecValTok{1}\NormalTok{, }\OtherTok{TRUE} \OperatorTok{~}\StringTok{ }\DecValTok{0}\NormalTok{)}
\NormalTok{  \}) }\OperatorTok
\StringTok{  }\NormalTok{dplyr}\OperatorTok{::}\KeywordTok{mutate_at}\NormalTok{(dplyr}\OperatorTok{::}\KeywordTok{all_of}\NormalTok{(comorbidades1), }\ControlFlowTok{function}\NormalTok{(x) \{}
\NormalTok{    dplyr}\OperatorTok{::}\KeywordTok{case_when}\NormalTok{(x }\OperatorTok{==}\StringTok{ "1"} \OperatorTok{~}\StringTok{ }\DecValTok{1}\NormalTok{, x }\OperatorTok{==}\StringTok{ "2"} \OperatorTok{~}\StringTok{ }\DecValTok{0}\NormalTok{, }\OtherTok{TRUE} \OperatorTok{~}\StringTok{ }\OtherTok{NA_real_}\NormalTok{)}
\NormalTok{  \}) }\OperatorTok
\StringTok{  }\NormalTok{dplyr}\OperatorTok{::}\KeywordTok{mutate}\NormalTok{(}
    \DataTypeTok{cont_comorb =}\NormalTok{ CARDIOPATI_aux }\OperatorTok{+}\StringTok{ }\NormalTok{HEMATOLOGI_aux }\OperatorTok{+}\StringTok{  }\NormalTok{HEPATICA_aux }\OperatorTok{+}\StringTok{ }\NormalTok{ASMA_aux }\OperatorTok{+}\StringTok{ }
\StringTok{      }\NormalTok{DIABETES_aux }\OperatorTok{+}\StringTok{ }\NormalTok{NEUROLOGIC_aux }\OperatorTok{+}\StringTok{ }\NormalTok{PNEUMOPATI_aux }\OperatorTok{+}\StringTok{ }\NormalTok{IMUNODEPRE_aux }\OperatorTok{+}\StringTok{ }
\StringTok{      }\NormalTok{RENAL_aux }\OperatorTok{+}\StringTok{ }\NormalTok{OBESIDADE_aux}
\NormalTok{  ) }\OperatorTok\StringTok{ }
\StringTok{  }\NormalTok{dplyr}\OperatorTok{::}\KeywordTok{mutate}\NormalTok{(}
    \DataTypeTok{num_comorb =}\NormalTok{ dplyr}\OperatorTok{::}\KeywordTok{case_when}\NormalTok{(}
      \KeywordTok{is.na}\NormalTok{(CARDIOPATI_aux1) }\OperatorTok{|}
\StringTok{        }\KeywordTok{is.na}\NormalTok{(HEMATOLOGI_aux1) }\OperatorTok{|}
\StringTok{        }\KeywordTok{is.na}\NormalTok{(HEPATICA_aux1) }\OperatorTok{|}
\StringTok{        }\KeywordTok{is.na}\NormalTok{(ASMA_aux1) }\OperatorTok{|}
\StringTok{        }\KeywordTok{is.na}\NormalTok{(DIABETES_aux1) }\OperatorTok{|}
\StringTok{        }\KeywordTok{is.na}\NormalTok{(NEUROLOGIC_aux1) }\OperatorTok{|}\StringTok{ }\KeywordTok{is.na}\NormalTok{(PNEUMOPATI_aux1) }\OperatorTok{|}
\StringTok{        }\KeywordTok{is.na}\NormalTok{(IMUNODEPRE_aux1) }\OperatorTok{|}
\StringTok{        }\KeywordTok{is.na}\NormalTok{(RENAL_aux1) }\OperatorTok{|}\StringTok{ }\KeywordTok{is.na}\NormalTok{(OBESIDADE_aux1) }\OperatorTok{~}\StringTok{ }\OtherTok{NA_real_}\NormalTok{,}
      \OtherTok{TRUE} \OperatorTok{~}\StringTok{ }\NormalTok{cont_comorb}
\NormalTok{    ),}
    \DataTypeTok{gr_comorb =}\NormalTok{ dplyr}\OperatorTok{::}\KeywordTok{case_when}\NormalTok{(}
\NormalTok{      num_comorb }\OperatorTok{==}\StringTok{ }\DecValTok{0} \OperatorTok{~}\StringTok{ }\DecValTok{0}\NormalTok{,}
\NormalTok{      num_comorb }\OperatorTok{==}\StringTok{ }\DecValTok{1} \OperatorTok{~}\StringTok{ }\DecValTok{1}\NormalTok{,}
\NormalTok{      num_comorb }\OperatorTok{==}\StringTok{ }\DecValTok{2} \OperatorTok{~}\StringTok{ }\DecValTok{1}\NormalTok{,}
\NormalTok{      num_comorb }\OperatorTok{>}\StringTok{  }\DecValTok{2} \OperatorTok{~}\StringTok{ }\DecValTok{2}\NormalTok{,}
      \OtherTok{TRUE} \OperatorTok{~}\StringTok{ }\OtherTok{NA_real_}
\NormalTok{    )}
\NormalTok{  )}

\CommentTok{# Comorbities group}
\NormalTok{data6}\OperatorTok{$}\NormalTok{gr_comorb <-}\StringTok{ }\KeywordTok{factor}\NormalTok{(data6}\OperatorTok{$}\NormalTok{gr_comorb,}
                          \DataTypeTok{levels =} \KeywordTok{c}\NormalTok{(}\DecValTok{0}\NormalTok{, }\DecValTok{1}\NormalTok{, }\DecValTok{2}\NormalTok{),}
                          \DataTypeTok{labels =} \KeywordTok{c}\NormalTok{(}\StringTok{"none"}\NormalTok{, }\StringTok{"1 or 2"}\NormalTok{, }\StringTok{">2"}\NormalTok{))}
\end{Highlighting}
\end{Shaded}

\begin{Shaded}
\begin{Highlighting}[]
\NormalTok{questionr}\OperatorTok{::}\KeywordTok{freq}\NormalTok{(}
\NormalTok{  data6}\OperatorTok{$}\NormalTok{gr_comorb,}
  \DataTypeTok{cum =} \OtherTok{FALSE}\NormalTok{,}
  \DataTypeTok{total =} \OtherTok{TRUE}\NormalTok{,}
  \DataTypeTok{na.last =} \OtherTok{FALSE}\NormalTok{,}
  \DataTypeTok{valid =} \OtherTok{FALSE}
\NormalTok{) }\OperatorTok
\StringTok{  }\NormalTok{knitr}\OperatorTok{::}\KeywordTok{kable}\NormalTok{(}\DataTypeTok{caption =} \StringTok{"Frequency table for comorbities group"}\NormalTok{, }\DataTypeTok{digits =} \DecValTok{2}\NormalTok{) }\OperatorTok
\StringTok{  }\KeywordTok{kable_styling}\NormalTok{(}\DataTypeTok{latex_options =} \StringTok{"hold_position"}\NormalTok{)}
\end{Highlighting}
\end{Shaded}

\begin{table}[!h]

\caption{\label{tab:unnamed-chunk-44}Frequency table for comorbities group}
\centering
\begin{tabular}[t]{l|r|r}
\hline
  & n & \%\\
\hline
none & 2933 & 7.2\\
\hline
1 or 2 & 8195 & 20.2\\
\hline
>2 & 830 & 2.0\\
\hline
NA & 28682 & 70.6\\
\hline
Total & 40640 & 100.0\\
\hline
\end{tabular}
\end{table}

Another variable of interest is metabolic syndrome defined here if one
has diabetes, heart disease and obesity. The variable name is
\texttt{gr\_sind\_met} with the categories ``yes'' and ``no''.

\begin{Shaded}
\begin{Highlighting}[]
\NormalTok{sind_met <-}\StringTok{  }\KeywordTok{c}\NormalTok{(}\StringTok{"CARDIOPATI_aux"}\NormalTok{,}
               \StringTok{"DIABETES_aux"}\NormalTok{, }\StringTok{"OBESIDADE_aux"}\NormalTok{)}

\NormalTok{sind_met1 <-}\StringTok{  }\KeywordTok{c}\NormalTok{(}\StringTok{"CARDIOPATI_aux1"}\NormalTok{,}
                \StringTok{"DIABETES_aux1"}\NormalTok{,}
                \StringTok{"OBESIDADE_aux1"}\NormalTok{)}

\NormalTok{data6 <-}
\StringTok{  }\KeywordTok{mutate}\NormalTok{(}
\NormalTok{    data6,}
    \DataTypeTok{CARDIOPATI_aux =}\NormalTok{ CARDIOPATI,}
    \DataTypeTok{DIABETES_aux =}\NormalTok{ DIABETES,}
    \DataTypeTok{OBESIDADE_aux =}\NormalTok{ OBESIDADE}
\NormalTok{  )}

\NormalTok{data6 <-}
\StringTok{  }\KeywordTok{mutate}\NormalTok{(}
\NormalTok{    data6,}
    \DataTypeTok{CARDIOPATI_aux1 =}\NormalTok{ CARDIOPATI,}
    \DataTypeTok{DIABETES_aux1 =}\NormalTok{ DIABETES,}
    \DataTypeTok{OBESIDADE_aux1 =}\NormalTok{ OBESIDADE}
\NormalTok{  )}

\NormalTok{data6 <-}\StringTok{  }\NormalTok{data6 }\OperatorTok
\StringTok{  }\KeywordTok{mutate_at}\NormalTok{(}\KeywordTok{all_of}\NormalTok{(sind_met), }\ControlFlowTok{function}\NormalTok{(x) \{}
    \KeywordTok{case_when}\NormalTok{(x }\OperatorTok{==}\StringTok{ "1"} \OperatorTok{~}\StringTok{ }\DecValTok{1}\NormalTok{, }\OtherTok{TRUE} \OperatorTok{~}\StringTok{ }\DecValTok{0}\NormalTok{)}
\NormalTok{  \}) }\OperatorTok
\StringTok{  }\KeywordTok{mutate_at}\NormalTok{(}\KeywordTok{all_of}\NormalTok{(sind_met1), }\ControlFlowTok{function}\NormalTok{(x) \{}
    \KeywordTok{case_when}\NormalTok{(x }\OperatorTok{==}\StringTok{ "1"} \OperatorTok{~}\StringTok{ }\DecValTok{1}\NormalTok{, x }\OperatorTok{==}\StringTok{ "2"} \OperatorTok{~}\StringTok{ }\DecValTok{0}\NormalTok{, }\OtherTok{TRUE} \OperatorTok{~}\StringTok{ }\OtherTok{NA_real_}\NormalTok{)}
\NormalTok{  \}) }\OperatorTok
\StringTok{  }\KeywordTok{mutate}\NormalTok{(}\DataTypeTok{cont_sind_met =}\NormalTok{ CARDIOPATI_aux }\OperatorTok{+}\StringTok{ }\NormalTok{DIABETES_aux }\OperatorTok{+}\StringTok{ }\NormalTok{OBESIDADE_aux) }\OperatorTok
\StringTok{  }\KeywordTok{mutate}\NormalTok{(}
    \DataTypeTok{num_sind_met =} \KeywordTok{case_when}\NormalTok{(}
      \KeywordTok{is.na}\NormalTok{(CARDIOPATI_aux1) }\OperatorTok{|}
\StringTok{        }\KeywordTok{is.na}\NormalTok{(DIABETES_aux1) }\OperatorTok{|}\StringTok{ }\KeywordTok{is.na}\NormalTok{(OBESIDADE_aux1) }\OperatorTok{~}\StringTok{ }\OtherTok{NA_real_}\NormalTok{,}
      \OtherTok{TRUE} \OperatorTok{~}\StringTok{ }\NormalTok{cont_sind_met}
\NormalTok{    ),}
    \DataTypeTok{gr_sind_met =} \KeywordTok{case_when}\NormalTok{(}
\NormalTok{      num_sind_met }\OperatorTok{==}\StringTok{ }\DecValTok{0} \OperatorTok{~}\StringTok{ }\DecValTok{0}\NormalTok{,}
\NormalTok{      num_sind_met }\OperatorTok{==}\StringTok{ }\DecValTok{1} \OperatorTok{~}\StringTok{ }\DecValTok{0}\NormalTok{,}
\NormalTok{      num_sind_met }\OperatorTok{==}\StringTok{ }\DecValTok{2} \OperatorTok{~}\StringTok{ }\DecValTok{0}\NormalTok{,}
\NormalTok{      num_sind_met }\OperatorTok{==}\StringTok{ }\DecValTok{3} \OperatorTok{~}\StringTok{ }\DecValTok{1}\NormalTok{,}
      \OtherTok{TRUE} \OperatorTok{~}\StringTok{ }\OtherTok{NA_real_}
\NormalTok{    )}
\NormalTok{  )}

\CommentTok{#metabolic syndrome indicator}
\NormalTok{data6}\OperatorTok{$}\NormalTok{gr_sind_met <-}\StringTok{ }\KeywordTok{factor}\NormalTok{(data6}\OperatorTok{$}\NormalTok{gr_sind_met,}
                            \DataTypeTok{levels =} \KeywordTok{c}\NormalTok{(}\DecValTok{1}\NormalTok{, }\DecValTok{0}\NormalTok{),}
                            \DataTypeTok{labels =} \KeywordTok{c}\NormalTok{(}\StringTok{"yes"}\NormalTok{, }\StringTok{"no"}\NormalTok{))}
\end{Highlighting}
\end{Shaded}

\begin{Shaded}
\begin{Highlighting}[]
\NormalTok{questionr}\OperatorTok{::}\KeywordTok{freq}\NormalTok{(}
\NormalTok{  data6}\OperatorTok{$}\NormalTok{gr_sind_met,}
  \DataTypeTok{cum =} \OtherTok{FALSE}\NormalTok{,}
  \DataTypeTok{total =} \OtherTok{TRUE}\NormalTok{,}
  \DataTypeTok{na.last =} \OtherTok{FALSE}\NormalTok{,}
  \DataTypeTok{valid =} \OtherTok{FALSE}
\NormalTok{) }\OperatorTok
\StringTok{  }\NormalTok{knitr}\OperatorTok{::}\KeywordTok{kable}\NormalTok{(}\DataTypeTok{caption =} \StringTok{"Frequency table for metabolic syndrome"}\NormalTok{, }\DataTypeTok{digits =} \DecValTok{2}\NormalTok{) }\OperatorTok
\StringTok{  }\KeywordTok{kable_styling}\NormalTok{(}\DataTypeTok{latex_options =} \StringTok{"hold_position"}\NormalTok{)}
\end{Highlighting}
\end{Shaded}

\begin{table}[!h]

\caption{\label{tab:unnamed-chunk-46}Frequency table for metabolic syndrome}
\centering
\begin{tabular}[t]{l|r|r}
\hline
  & n & \%\\
\hline
yes & 436 & 1.1\\
\hline
no & 12073 & 29.7\\
\hline
NA & 28131 & 69.2\\
\hline
Total & 40640 & 100.0\\
\hline
\end{tabular}
\end{table}

\hypertarget{symptom-variables-and-indicator-of-hospital-acquired-infection}{%
\subsection{3.2 Symptom variables and indicator of hospital-acquired
infection}\label{symptom-variables-and-indicator-of-hospital-acquired-infection}}

For the indicator of a case arising from an infection acquired in the
hospital (\texttt{NOSOCOMIAL}), the categories are 1-yes, 2-no and
9-ignored.

\begin{Shaded}
\begin{Highlighting}[]
\NormalTok{questionr}\OperatorTok{::}\KeywordTok{freq}\NormalTok{(}
\NormalTok{  data6}\OperatorTok{$}\NormalTok{NOSOCOMIAL,}
  \DataTypeTok{cum =} \OtherTok{FALSE}\NormalTok{,}
  \DataTypeTok{total =} \OtherTok{TRUE}\NormalTok{,}
  \DataTypeTok{na.last =} \OtherTok{FALSE}\NormalTok{,}
  \DataTypeTok{valid =} \OtherTok{FALSE}
\NormalTok{) }\OperatorTok\StringTok{ }
\StringTok{  }\NormalTok{knitr}\OperatorTok{::}\KeywordTok{kable}\NormalTok{(}\DataTypeTok{caption =} \StringTok{"Frequency table for hospital-acquired infection"}\NormalTok{, }\DataTypeTok{digits =} \DecValTok{2}\NormalTok{) }\OperatorTok
\StringTok{  }\KeywordTok{kable_styling}\NormalTok{(}\DataTypeTok{latex_options =} \StringTok{"hold_position"}\NormalTok{)}
\end{Highlighting}
\end{Shaded}

\begin{table}[!h]

\caption{\label{tab:unnamed-chunk-47}Frequency table for hospital-acquired infection}
\centering
\begin{tabular}[t]{l|r|r}
\hline
  & n & \%\\
\hline
1 & 831 & 2.0\\
\hline
2 & 29891 & 73.6\\
\hline
9 & 2802 & 6.9\\
\hline
NA & 7116 & 17.5\\
\hline
Total & 40640 & 100.0\\
\hline
\end{tabular}
\end{table}

We will now label this variable, creating the variable
\texttt{inf\_inter}, considering only the valid categories.

\begin{Shaded}
\begin{Highlighting}[]
\NormalTok{data6}\OperatorTok{$}\NormalTok{inf_inter <-}\StringTok{ }\KeywordTok{factor}\NormalTok{(data6}\OperatorTok{$}\NormalTok{NOSOCOMIAL,}
                          \DataTypeTok{levels =} \KeywordTok{c}\NormalTok{(}\StringTok{"1"}\NormalTok{, }\StringTok{"2"}\NormalTok{),}
                          \DataTypeTok{labels =} \KeywordTok{c}\NormalTok{(}\StringTok{"yes"}\NormalTok{, }\StringTok{"no"}\NormalTok{))}
\end{Highlighting}
\end{Shaded}

The symptoms are fever, cough, sore throat, dyspnoea, vomiting,
abdominal pain, fatigue, respiratory distress, saturation, diarrhea,
olfactory loss and loss of taste. In the original dataset they are
\texttt{FEBRE}, \texttt{TOSSE}, \texttt{GARGANTA}, \texttt{DISPNEIA},
\texttt{VOMITO}, \texttt{DOR\_ABD}, \texttt{FADIGA},
\texttt{DESC\_RESP}, \texttt{SATURACAO}, \texttt{DIARREIA},
\texttt{PERD\_OLFT}, \texttt{PERD\_PALA}, respectively. The categories
of these variables are 1-yes, 2-no and 9-ignored.

\begin{Shaded}
\begin{Highlighting}[]
\NormalTok{questionr}\OperatorTok{::}\KeywordTok{freq}\NormalTok{(}
\NormalTok{  data6}\OperatorTok{$}\NormalTok{FEBRE,}
  \DataTypeTok{cum =} \OtherTok{FALSE}\NormalTok{,}
  \DataTypeTok{total =} \OtherTok{TRUE}\NormalTok{,}
  \DataTypeTok{na.last =} \OtherTok{FALSE}\NormalTok{,}
  \DataTypeTok{valid =} \OtherTok{FALSE}
\NormalTok{) }\OperatorTok
\StringTok{  }\NormalTok{knitr}\OperatorTok{::}\KeywordTok{kable}\NormalTok{(}\DataTypeTok{caption =} \StringTok{"Frequency table for fever indicator"}\NormalTok{, }\DataTypeTok{digits =} \DecValTok{2}\NormalTok{) }\OperatorTok
\StringTok{  }\KeywordTok{kable_styling}\NormalTok{(}\DataTypeTok{latex_options =} \StringTok{"hold_position"}\NormalTok{)}
\end{Highlighting}
\end{Shaded}

\begin{table}[!h]

\caption{\label{tab:unnamed-chunk-49}Frequency table for fever indicator}
\centering
\begin{tabular}[t]{l|r|r}
\hline
  & n & \%\\
\hline
1 & 26759 & 65.8\\
\hline
2 & 9612 & 23.7\\
\hline
9 & 360 & 0.9\\
\hline
NA & 3909 & 9.6\\
\hline
Total & 40640 & 100.0\\
\hline
\end{tabular}
\end{table}

\begin{Shaded}
\begin{Highlighting}[]
\NormalTok{questionr}\OperatorTok{::}\KeywordTok{freq}\NormalTok{(}
\NormalTok{  data6}\OperatorTok{$}\NormalTok{TOSSE,}
  \DataTypeTok{cum =} \OtherTok{FALSE}\NormalTok{,}
  \DataTypeTok{total =} \OtherTok{TRUE}\NormalTok{,}
  \DataTypeTok{na.last =} \OtherTok{FALSE}\NormalTok{,}
  \DataTypeTok{valid =} \OtherTok{FALSE}
\NormalTok{) }\OperatorTok
\StringTok{  }\NormalTok{knitr}\OperatorTok{::}\KeywordTok{kable}\NormalTok{(}\DataTypeTok{caption =} \StringTok{"Frequency table for cough indicator"}\NormalTok{, }\DataTypeTok{digits =} \DecValTok{2}\NormalTok{) }\OperatorTok
\StringTok{  }\KeywordTok{kable_styling}\NormalTok{(}\DataTypeTok{latex_options =} \StringTok{"hold_position"}\NormalTok{)}
\end{Highlighting}
\end{Shaded}

\begin{table}[!h]

\caption{\label{tab:unnamed-chunk-50}Frequency table for cough indicator}
\centering
\begin{tabular}[t]{l|r|r}
\hline
  & n & \%\\
\hline
1 & 30188 & 74.3\\
\hline
2 & 6913 & 17.0\\
\hline
9 & 296 & 0.7\\
\hline
NA & 3243 & 8.0\\
\hline
Total & 40640 & 100.0\\
\hline
\end{tabular}
\end{table}

\begin{Shaded}
\begin{Highlighting}[]
\NormalTok{questionr}\OperatorTok{::}\KeywordTok{freq}\NormalTok{(}
\NormalTok{  data6}\OperatorTok{$}\NormalTok{GARGANTA,}
  \DataTypeTok{cum =} \OtherTok{FALSE}\NormalTok{,}
  \DataTypeTok{total =} \OtherTok{TRUE}\NormalTok{,}
  \DataTypeTok{na.last =} \OtherTok{FALSE}\NormalTok{,}
  \DataTypeTok{valid =} \OtherTok{FALSE}
\NormalTok{) }\OperatorTok
\StringTok{  }\NormalTok{knitr}\OperatorTok{::}\KeywordTok{kable}\NormalTok{(}\DataTypeTok{caption =} \StringTok{"Frequency table for sore throat indicator"}\NormalTok{, }\DataTypeTok{digits =} \DecValTok{2}\NormalTok{) }\OperatorTok
\StringTok{  }\KeywordTok{kable_styling}\NormalTok{(}\DataTypeTok{latex_options =} \StringTok{"hold_position"}\NormalTok{)}
\end{Highlighting}
\end{Shaded}

\begin{table}[!h]

\caption{\label{tab:unnamed-chunk-51}Frequency table for sore throat indicator}
\centering
\begin{tabular}[t]{l|r|r}
\hline
  & n & \%\\
\hline
1 & 9734 & 24.0\\
\hline
2 & 21740 & 53.5\\
\hline
9 & 612 & 1.5\\
\hline
NA & 8554 & 21.0\\
\hline
Total & 40640 & 100.0\\
\hline
\end{tabular}
\end{table}

\begin{Shaded}
\begin{Highlighting}[]
\NormalTok{questionr}\OperatorTok{::}\KeywordTok{freq}\NormalTok{(}
\NormalTok{  data6}\OperatorTok{$}\NormalTok{DISPNEIA,}
  \DataTypeTok{cum =} \OtherTok{FALSE}\NormalTok{,}
  \DataTypeTok{total =} \OtherTok{TRUE}\NormalTok{,}
  \DataTypeTok{na.last =} \OtherTok{FALSE}\NormalTok{,}
  \DataTypeTok{valid =} \OtherTok{FALSE}
\NormalTok{) }\OperatorTok
\StringTok{  }\NormalTok{knitr}\OperatorTok{::}\KeywordTok{kable}\NormalTok{(}\DataTypeTok{caption =} \StringTok{"Frequency table for dyspnea indicator"}\NormalTok{, }\DataTypeTok{digits =} \DecValTok{2}\NormalTok{) }\OperatorTok
\StringTok{  }\KeywordTok{kable_styling}\NormalTok{(}\DataTypeTok{latex_options =} \StringTok{"hold_position"}\NormalTok{)}
\end{Highlighting}
\end{Shaded}

\begin{table}[!h]

\caption{\label{tab:unnamed-chunk-52}Frequency table for dyspnea indicator}
\centering
\begin{tabular}[t]{l|r|r}
\hline
  & n & \%\\
\hline
1 & 27276 & 67.1\\
\hline
2 & 8996 & 22.1\\
\hline
9 & 295 & 0.7\\
\hline
NA & 4073 & 10.0\\
\hline
Total & 40640 & 100.0\\
\hline
\end{tabular}
\end{table}

\begin{Shaded}
\begin{Highlighting}[]
\NormalTok{questionr}\OperatorTok{::}\KeywordTok{freq}\NormalTok{(}
\NormalTok{  data6}\OperatorTok{$}\NormalTok{VOMITO,}
  \DataTypeTok{cum =} \OtherTok{FALSE}\NormalTok{,}
  \DataTypeTok{total =} \OtherTok{TRUE}\NormalTok{,}
  \DataTypeTok{na.last =} \OtherTok{FALSE}\NormalTok{,}
  \DataTypeTok{valid =} \OtherTok{FALSE}
\NormalTok{) }\OperatorTok
\StringTok{  }\NormalTok{knitr}\OperatorTok{::}\KeywordTok{kable}\NormalTok{(}\DataTypeTok{caption =} \StringTok{"Frequency table for vomiting"}\NormalTok{, }\DataTypeTok{digits =} \DecValTok{2}\NormalTok{) }\OperatorTok
\StringTok{  }\KeywordTok{kable_styling}\NormalTok{(}\DataTypeTok{latex_options =} \StringTok{"hold_position"}\NormalTok{)}
\end{Highlighting}
\end{Shaded}

\begin{table}[!h]

\caption{\label{tab:unnamed-chunk-53}Frequency table for vomiting}
\centering
\begin{tabular}[t]{l|r|r}
\hline
  & n & \%\\
\hline
1 & 4535 & 11.2\\
\hline
2 & 25807 & 63.5\\
\hline
9 & 681 & 1.7\\
\hline
NA & 9617 & 23.7\\
\hline
Total & 40640 & 100.0\\
\hline
\end{tabular}
\end{table}

\begin{Shaded}
\begin{Highlighting}[]
\NormalTok{questionr}\OperatorTok{::}\KeywordTok{freq}\NormalTok{(}
\NormalTok{  data6}\OperatorTok{$}\NormalTok{DOR_ABD,}
  \DataTypeTok{cum =} \OtherTok{FALSE}\NormalTok{,}
  \DataTypeTok{total =} \OtherTok{TRUE}\NormalTok{,}
  \DataTypeTok{na.last =} \OtherTok{FALSE}\NormalTok{,}
  \DataTypeTok{valid =} \OtherTok{FALSE}
\NormalTok{) }\OperatorTok
\StringTok{  }\NormalTok{knitr}\OperatorTok{::}\KeywordTok{kable}\NormalTok{(}\DataTypeTok{caption =} \StringTok{"Frequency table for abdominal pain"}\NormalTok{, }\DataTypeTok{digits =} \DecValTok{2}\NormalTok{) }\OperatorTok
\StringTok{  }\KeywordTok{kable_styling}\NormalTok{(}\DataTypeTok{latex_options =} \StringTok{"hold_position"}\NormalTok{)}
\end{Highlighting}
\end{Shaded}

\begin{table}[!h]

\caption{\label{tab:unnamed-chunk-54}Frequency table for abdominal pain}
\centering
\begin{tabular}[t]{l|r|r}
\hline
  & n & \%\\
\hline
1 & 1484 & 3.7\\
\hline
2 & 14291 & 35.2\\
\hline
9 & 493 & 1.2\\
\hline
NA & 24372 & 60.0\\
\hline
Total & 40640 & 100.0\\
\hline
\end{tabular}
\end{table}

\begin{Shaded}
\begin{Highlighting}[]
\NormalTok{questionr}\OperatorTok{::}\KeywordTok{freq}\NormalTok{(}
\NormalTok{  data6}\OperatorTok{$}\NormalTok{FADIGA,}
  \DataTypeTok{cum =} \OtherTok{FALSE}\NormalTok{,}
  \DataTypeTok{total =} \OtherTok{TRUE}\NormalTok{,}
  \DataTypeTok{na.last =} \OtherTok{FALSE}\NormalTok{,}
  \DataTypeTok{valid =} \OtherTok{FALSE}
\NormalTok{) }\OperatorTok
\StringTok{  }\NormalTok{knitr}\OperatorTok{::}\KeywordTok{kable}\NormalTok{(}\DataTypeTok{caption =} \StringTok{"Frequency table for fatigue"}\NormalTok{, }\DataTypeTok{digits =} \DecValTok{2}\NormalTok{) }\OperatorTok
\StringTok{  }\KeywordTok{kable_styling}\NormalTok{(}\DataTypeTok{latex_options =} \StringTok{"hold_position"}\NormalTok{)}
\end{Highlighting}
\end{Shaded}

\begin{table}[!h]

\caption{\label{tab:unnamed-chunk-55}Frequency table for fatigue}
\centering
\begin{tabular}[t]{l|r|r}
\hline
  & n & \%\\
\hline
1 & 4693 & 11.5\\
\hline
2 & 11523 & 28.4\\
\hline
9 & 497 & 1.2\\
\hline
NA & 23927 & 58.9\\
\hline
Total & 40640 & 100.0\\
\hline
\end{tabular}
\end{table}

\begin{Shaded}
\begin{Highlighting}[]
\NormalTok{questionr}\OperatorTok{::}\KeywordTok{freq}\NormalTok{(}
\NormalTok{  data6}\OperatorTok{$}\NormalTok{DESC_RESP,}
  \DataTypeTok{cum =} \OtherTok{FALSE}\NormalTok{,}
  \DataTypeTok{total =} \OtherTok{TRUE}\NormalTok{,}
  \DataTypeTok{na.last =} \OtherTok{FALSE}\NormalTok{,}
  \DataTypeTok{valid =} \OtherTok{FALSE}
\NormalTok{) }\OperatorTok
\StringTok{  }\NormalTok{knitr}\OperatorTok{::}\KeywordTok{kable}\NormalTok{(}\DataTypeTok{caption =} \StringTok{"Frequency table for respiratory discomfort"}\NormalTok{, }\DataTypeTok{digits =} \DecValTok{2}\NormalTok{) }\OperatorTok
\StringTok{  }\KeywordTok{kable_styling}\NormalTok{(}\DataTypeTok{latex_options =} \StringTok{"hold_position"}\NormalTok{)}
\end{Highlighting}
\end{Shaded}

\begin{table}[!h]

\caption{\label{tab:unnamed-chunk-56}Frequency table for respiratory discomfort}
\centering
\begin{tabular}[t]{l|r|r}
\hline
  & n & \%\\
\hline
1 & 21873 & 53.8\\
\hline
2 & 12150 & 29.9\\
\hline
9 & 423 & 1.0\\
\hline
NA & 6194 & 15.2\\
\hline
Total & 40640 & 100.0\\
\hline
\end{tabular}
\end{table}

\begin{Shaded}
\begin{Highlighting}[]
\NormalTok{questionr}\OperatorTok{::}\KeywordTok{freq}\NormalTok{(}
\NormalTok{  data6}\OperatorTok{$}\NormalTok{SATURACAO,}
  \DataTypeTok{cum =} \OtherTok{FALSE}\NormalTok{,}
  \DataTypeTok{total =} \OtherTok{TRUE}\NormalTok{,}
  \DataTypeTok{na.last =} \OtherTok{FALSE}\NormalTok{,}
  \DataTypeTok{valid =} \OtherTok{FALSE}
\NormalTok{) }\OperatorTok
\StringTok{  }\NormalTok{knitr}\OperatorTok{::}\KeywordTok{kable}\NormalTok{(}\DataTypeTok{caption =} \StringTok{"Frequency table for saturation"}\NormalTok{, }\DataTypeTok{digits =} \DecValTok{2}\NormalTok{) }\OperatorTok
\StringTok{  }\KeywordTok{kable_styling}\NormalTok{(}\DataTypeTok{latex_options =} \StringTok{"hold_position"}\NormalTok{)}
\end{Highlighting}
\end{Shaded}

\begin{table}[!h]

\caption{\label{tab:unnamed-chunk-57}Frequency table for saturation}
\centering
\begin{tabular}[t]{l|r|r}
\hline
  & n & \%\\
\hline
1 & 18260 & 44.9\\
\hline
2 & 15222 & 37.5\\
\hline
9 & 529 & 1.3\\
\hline
NA & 6629 & 16.3\\
\hline
Total & 40640 & 100.0\\
\hline
\end{tabular}
\end{table}

\begin{Shaded}
\begin{Highlighting}[]
\NormalTok{questionr}\OperatorTok{::}\KeywordTok{freq}\NormalTok{(}
\NormalTok{  data6}\OperatorTok{$}\NormalTok{DIARREIA,}
  \DataTypeTok{cum =} \OtherTok{FALSE}\NormalTok{,}
  \DataTypeTok{total =} \OtherTok{TRUE}\NormalTok{,}
  \DataTypeTok{na.last =} \OtherTok{FALSE}\NormalTok{,}
  \DataTypeTok{valid =} \OtherTok{FALSE}
\NormalTok{) }\OperatorTok
\StringTok{  }\NormalTok{knitr}\OperatorTok{::}\KeywordTok{kable}\NormalTok{(}\DataTypeTok{caption =} \StringTok{"Frequency table for diarrhea"}\NormalTok{, }\DataTypeTok{digits =} \DecValTok{2}\NormalTok{) }\OperatorTok
\StringTok{  }\KeywordTok{kable_styling}\NormalTok{(}\DataTypeTok{latex_options =} \StringTok{"hold_position"}\NormalTok{)}
\end{Highlighting}
\end{Shaded}

\begin{table}[!h]

\caption{\label{tab:unnamed-chunk-58}Frequency table for diarrhea}
\centering
\begin{tabular}[t]{l|r|r}
\hline
  & n & \%\\
\hline
1 & 6787 & 16.7\\
\hline
2 & 24120 & 59.4\\
\hline
9 & 621 & 1.5\\
\hline
NA & 9112 & 22.4\\
\hline
Total & 40640 & 100.0\\
\hline
\end{tabular}
\end{table}

\begin{Shaded}
\begin{Highlighting}[]
\NormalTok{questionr}\OperatorTok{::}\KeywordTok{freq}\NormalTok{(}
\NormalTok{  data6}\OperatorTok{$}\NormalTok{PERD_OLFT,}
  \DataTypeTok{cum =} \OtherTok{FALSE}\NormalTok{,}
  \DataTypeTok{total =} \OtherTok{TRUE}\NormalTok{,}
  \DataTypeTok{na.last =} \OtherTok{FALSE}\NormalTok{,}
  \DataTypeTok{valid =} \OtherTok{FALSE}
\NormalTok{) }\OperatorTok
\StringTok{  }\NormalTok{knitr}\OperatorTok{::}\KeywordTok{kable}\NormalTok{(}\DataTypeTok{caption =} \StringTok{"Frequency table for olfactory loss"}\NormalTok{, }\DataTypeTok{digits =} \DecValTok{2}\NormalTok{) }\OperatorTok
\StringTok{  }\KeywordTok{kable_styling}\NormalTok{(}\DataTypeTok{latex_options =} \StringTok{"hold_position"}\NormalTok{)}
\end{Highlighting}
\end{Shaded}

\begin{table}[!h]

\caption{\label{tab:unnamed-chunk-59}Frequency table for olfactory loss}
\centering
\begin{tabular}[t]{l|r|r}
\hline
  & n & \%\\
\hline
1 & 3602 & 8.9\\
\hline
2 & 12600 & 31.0\\
\hline
9 & 542 & 1.3\\
\hline
NA & 23896 & 58.8\\
\hline
Total & 40640 & 100.0\\
\hline
\end{tabular}
\end{table}

\begin{Shaded}
\begin{Highlighting}[]
\NormalTok{questionr}\OperatorTok{::}\KeywordTok{freq}\NormalTok{(}
\NormalTok{  data6}\OperatorTok{$}\NormalTok{PERD_PALA,}
  \DataTypeTok{cum =} \OtherTok{FALSE}\NormalTok{,}
  \DataTypeTok{total =} \OtherTok{TRUE}\NormalTok{,}
  \DataTypeTok{na.last =} \OtherTok{FALSE}\NormalTok{,}
  \DataTypeTok{valid =} \OtherTok{FALSE}
\NormalTok{) }\OperatorTok
\StringTok{  }\NormalTok{knitr}\OperatorTok{::}\KeywordTok{kable}\NormalTok{(}\DataTypeTok{caption =} \StringTok{"Frequency table for loss of taste"}\NormalTok{, }\DataTypeTok{digits =} \DecValTok{2}\NormalTok{) }\OperatorTok
\StringTok{  }\KeywordTok{kable_styling}\NormalTok{(}\DataTypeTok{latex_options =} \StringTok{"hold_position"}\NormalTok{)}
\end{Highlighting}
\end{Shaded}

\begin{table}[!h]

\caption{\label{tab:unnamed-chunk-60}Frequency table for loss of taste}
\centering
\begin{tabular}[t]{l|r|r}
\hline
  & n & \%\\
\hline
1 & 3463 & 8.5\\
\hline
2 & 12640 & 31.1\\
\hline
9 & 550 & 1.4\\
\hline
NA & 23987 & 59.0\\
\hline
Total & 40640 & 100.0\\
\hline
\end{tabular}
\end{table}

We will now label the sympthoms variables by considering only the valid
categories and creating the variable \texttt{febre}, \texttt{tosse},
\texttt{garganta}, \texttt{dispneia}, \texttt{vomito},
\texttt{dor\_abd}, \texttt{fadiga}, \texttt{desc\_resp},
\texttt{saturacao}, \texttt{diarreia}, \texttt{perd\_olft} and
\texttt{perd\_pala} that represent fever, cough, sore throat, dyspnoea,
vomiting, abdominal pain, fatigue, respiratory distress, saturation,
diarrhea, olfactory loss and loss of taste, respectively.

\begin{Shaded}
\begin{Highlighting}[]
\CommentTok{#fever}
\NormalTok{data6}\OperatorTok{$}\NormalTok{febre <-}\StringTok{ }\KeywordTok{factor}\NormalTok{(data6}\OperatorTok{$}\NormalTok{FEBRE,}
                       \DataTypeTok{levels =} \KeywordTok{c}\NormalTok{(}\StringTok{"1"}\NormalTok{, }\StringTok{"2"}\NormalTok{),}
                       \DataTypeTok{labels =} \KeywordTok{c}\NormalTok{(}\StringTok{"yes"}\NormalTok{, }\StringTok{"no"}\NormalTok{))}
\end{Highlighting}
\end{Shaded}

\begin{Shaded}
\begin{Highlighting}[]
\CommentTok{#cough}
\NormalTok{data6}\OperatorTok{$}\NormalTok{tosse <-}\StringTok{ }\KeywordTok{factor}\NormalTok{(data6}\OperatorTok{$}\NormalTok{TOSSE,}
                       \DataTypeTok{levels =} \KeywordTok{c}\NormalTok{(}\StringTok{"1"}\NormalTok{, }\StringTok{"2"}\NormalTok{),}
                       \DataTypeTok{labels =} \KeywordTok{c}\NormalTok{(}\StringTok{"yes"}\NormalTok{, }\StringTok{"no"}\NormalTok{))}
\end{Highlighting}
\end{Shaded}

\begin{Shaded}
\begin{Highlighting}[]
\CommentTok{#sore throat}
\NormalTok{data6}\OperatorTok{$}\NormalTok{garganta <-}\StringTok{ }\KeywordTok{factor}\NormalTok{(data6}\OperatorTok{$}\NormalTok{GARGANTA,}
                          \DataTypeTok{levels =} \KeywordTok{c}\NormalTok{(}\StringTok{"1"}\NormalTok{, }\StringTok{"2"}\NormalTok{),}
                          \DataTypeTok{labels =} \KeywordTok{c}\NormalTok{(}\StringTok{"yes"}\NormalTok{, }\StringTok{"no"}\NormalTok{))}
\end{Highlighting}
\end{Shaded}

\begin{Shaded}
\begin{Highlighting}[]
\CommentTok{#dyspnoea}
\NormalTok{data6}\OperatorTok{$}\NormalTok{dispneia <-}\StringTok{ }\KeywordTok{factor}\NormalTok{(data6}\OperatorTok{$}\NormalTok{DISPNEIA, }
                     \DataTypeTok{levels =} \KeywordTok{c}\NormalTok{(}\StringTok{"1"}\NormalTok{, }\StringTok{"2"}\NormalTok{), }
                     \DataTypeTok{labels =} \KeywordTok{c}\NormalTok{(}\StringTok{"yes"}\NormalTok{, }\StringTok{"no"}\NormalTok{))}
\end{Highlighting}
\end{Shaded}

\begin{Shaded}
\begin{Highlighting}[]
\CommentTok{#vomiting}
\NormalTok{data6}\OperatorTok{$}\NormalTok{vomito <-}\StringTok{ }\KeywordTok{factor}\NormalTok{(data6}\OperatorTok{$}\NormalTok{VOMITO,}
                        \DataTypeTok{levels =} \KeywordTok{c}\NormalTok{(}\StringTok{"1"}\NormalTok{, }\StringTok{"2"}\NormalTok{),}
                        \DataTypeTok{labels =} \KeywordTok{c}\NormalTok{(}\StringTok{"yes"}\NormalTok{, }\StringTok{"no"}\NormalTok{))}
\end{Highlighting}
\end{Shaded}

\begin{Shaded}
\begin{Highlighting}[]
\CommentTok{#abdominal pain}
\NormalTok{data6}\OperatorTok{$}\NormalTok{dor_abd <-}\StringTok{ }\KeywordTok{factor}\NormalTok{(data6}\OperatorTok{$}\NormalTok{DOR_ABD,}
                         \DataTypeTok{levels =} \KeywordTok{c}\NormalTok{(}\StringTok{"1"}\NormalTok{, }\StringTok{"2"}\NormalTok{),}
                         \DataTypeTok{labels =} \KeywordTok{c}\NormalTok{(}\StringTok{"yes"}\NormalTok{, }\StringTok{"no"}\NormalTok{))}
\end{Highlighting}
\end{Shaded}

\begin{Shaded}
\begin{Highlighting}[]
\CommentTok{#fatigue}
\NormalTok{data6}\OperatorTok{$}\NormalTok{fadiga <-}\StringTok{ }\KeywordTok{factor}\NormalTok{(data6}\OperatorTok{$}\NormalTok{FADIGA,}
                        \DataTypeTok{levels =} \KeywordTok{c}\NormalTok{(}\StringTok{"1"}\NormalTok{, }\StringTok{"2"}\NormalTok{),}
                        \DataTypeTok{labels =} \KeywordTok{c}\NormalTok{(}\StringTok{"yes"}\NormalTok{, }\StringTok{"no"}\NormalTok{))}
\end{Highlighting}
\end{Shaded}

\begin{Shaded}
\begin{Highlighting}[]
\CommentTok{#respiratory distress}
\NormalTok{data6}\OperatorTok{$}\NormalTok{desc_resp <-}\StringTok{ }\KeywordTok{factor}\NormalTok{(data6}\OperatorTok{$}\NormalTok{DESC_RESP,}
                           \DataTypeTok{levels =} \KeywordTok{c}\NormalTok{(}\StringTok{"1"}\NormalTok{, }\StringTok{"2"}\NormalTok{),}
                           \DataTypeTok{labels =} \KeywordTok{c}\NormalTok{(}\StringTok{"yes"}\NormalTok{, }\StringTok{"no"}\NormalTok{))}
\end{Highlighting}
\end{Shaded}

\begin{Shaded}
\begin{Highlighting}[]
\CommentTok{#saturation}
\NormalTok{data6}\OperatorTok{$}\NormalTok{saturacao <-}\StringTok{ }\KeywordTok{factor}\NormalTok{(data6}\OperatorTok{$}\NormalTok{SATURACAO, }
                     \DataTypeTok{levels =} \KeywordTok{c}\NormalTok{(}\StringTok{"1"}\NormalTok{, }\StringTok{"2"}\NormalTok{), }
                     \DataTypeTok{labels =} \KeywordTok{c}\NormalTok{(}\StringTok{"yes"}\NormalTok{, }\StringTok{"no"}\NormalTok{))}
\end{Highlighting}
\end{Shaded}

\begin{Shaded}
\begin{Highlighting}[]
\CommentTok{#diarrhea}
\NormalTok{data6}\OperatorTok{$}\NormalTok{diarreia <-}\StringTok{ }\KeywordTok{factor}\NormalTok{(data6}\OperatorTok{$}\NormalTok{DIARREIA, }
                     \DataTypeTok{levels =} \KeywordTok{c}\NormalTok{(}\StringTok{"1"}\NormalTok{, }\StringTok{"2"}\NormalTok{), }
                     \DataTypeTok{labels =} \KeywordTok{c}\NormalTok{(}\StringTok{"yes"}\NormalTok{, }\StringTok{"no"}\NormalTok{))}
\end{Highlighting}
\end{Shaded}

\begin{Shaded}
\begin{Highlighting}[]
\CommentTok{#olfactory loss}
\NormalTok{data6}\OperatorTok{$}\NormalTok{perd_olft <-}\StringTok{ }\KeywordTok{factor}\NormalTok{(data6}\OperatorTok{$}\NormalTok{PERD_OLFT,}
                           \DataTypeTok{levels =} \KeywordTok{c}\NormalTok{(}\StringTok{"1"}\NormalTok{, }\StringTok{"2"}\NormalTok{),}
                           \DataTypeTok{labels =} \KeywordTok{c}\NormalTok{(}\StringTok{"yes"}\NormalTok{, }\StringTok{"no"}\NormalTok{))}
\end{Highlighting}
\end{Shaded}

\begin{Shaded}
\begin{Highlighting}[]
\CommentTok{#loss of taste}
\NormalTok{data6}\OperatorTok{$}\NormalTok{perd_pala <-}\StringTok{ }\KeywordTok{factor}\NormalTok{(data6}\OperatorTok{$}\NormalTok{PERD_PALA,}
                           \DataTypeTok{levels =} \KeywordTok{c}\NormalTok{(}\StringTok{"1"}\NormalTok{, }\StringTok{"2"}\NormalTok{),}
                           \DataTypeTok{labels =} \KeywordTok{c}\NormalTok{(}\StringTok{"yes"}\NormalTok{, }\StringTok{"no"}\NormalTok{))}
\end{Highlighting}
\end{Shaded}

Besides the indicator variable of each symptom, the variable group of
symptoms has three categories: ``none'', ``1 or 2'' and
``\textgreater2'' (\texttt{gr\_sintomas}) and another indicator variable
of at least one symptom (\texttt{sintomas\_SN}), with categories ``yes''
or ``no''. The symptoms are fever, cough, sore throat, dyspnoea,
respiratory distress, saturation, diarrhea, vomiting, abdominal pain,
fatigue, olfactory loss and loss of taste.

\begin{Shaded}
\begin{Highlighting}[]
\NormalTok{sintomas <-}
\StringTok{  }\KeywordTok{c}\NormalTok{(}
    \StringTok{"FEBRE_aux"}\NormalTok{,}
    \StringTok{"TOSSE_aux"}\NormalTok{,}
    \StringTok{"GARGANTA_aux"}\NormalTok{,}
    \StringTok{"DISPNEIA_aux"}\NormalTok{,}
    \StringTok{"DESC_RESP_aux"}\NormalTok{,}
    \StringTok{"SATURACAO_aux"}\NormalTok{,}
    \StringTok{"DIARREIA_aux"}\NormalTok{,}
    \StringTok{"VOMITO_aux"}\NormalTok{,}
    \StringTok{"DOR_ABD_aux"}\NormalTok{,}
    \StringTok{"FADIGA_aux"}\NormalTok{,}
    \StringTok{"PERD_OLFT_aux"}\NormalTok{,}
    \StringTok{"PERD_PALA_aux"}
\NormalTok{  )}

\NormalTok{sintomas1 <-}
\StringTok{  }\KeywordTok{c}\NormalTok{(}
    \StringTok{"FEBRE_aux1"}\NormalTok{,}
    \StringTok{"TOSSE_aux1"}\NormalTok{,}
    \StringTok{"GARGANTA_aux1"}\NormalTok{,}
    \StringTok{"DISPNEIA_aux1"}\NormalTok{,}
    \StringTok{"DESC_RESP_aux1"}\NormalTok{,}
    \StringTok{"SATURACAO_aux1"}\NormalTok{,}
    \StringTok{"DIARREIA_aux1"}\NormalTok{,}
    \StringTok{"VOMITO_aux1"}\NormalTok{,}
    \StringTok{"DOR_ABD_aux1"}\NormalTok{,}
    \StringTok{"FADIGA_aux1"}\NormalTok{,}
    \StringTok{"PERD_OLFT_aux1"}\NormalTok{,}
    \StringTok{"PERD_PALA_aux1"}
\NormalTok{  )}

\NormalTok{data6 <-}
\StringTok{  }\KeywordTok{mutate}\NormalTok{(}
\NormalTok{    data6,}
    \DataTypeTok{FEBRE_aux =}\NormalTok{ FEBRE,}
    \DataTypeTok{TOSSE_aux =}\NormalTok{ TOSSE,}
    \DataTypeTok{GARGANTA_aux =}\NormalTok{ GARGANTA,}
    \DataTypeTok{DISPNEIA_aux =}\NormalTok{ DISPNEIA,}
    \DataTypeTok{DESC_RESP_aux =}\NormalTok{ DESC_RESP,}
    \DataTypeTok{SATURACAO_aux =}\NormalTok{ SATURACAO,}
    \DataTypeTok{DIARREIA_aux =}\NormalTok{ DIARREIA,}
    \DataTypeTok{VOMITO_aux =}\NormalTok{ VOMITO,}
    \DataTypeTok{DOR_ABD_aux =}\NormalTok{ DOR_ABD,}
    \DataTypeTok{FADIGA_aux =}\NormalTok{ FADIGA,}
    \DataTypeTok{PERD_OLFT_aux =}\NormalTok{ PERD_OLFT,}
    \DataTypeTok{PERD_PALA_aux =}\NormalTok{ PERD_PALA}
\NormalTok{  )}

\NormalTok{data6 <-}
\StringTok{  }\KeywordTok{mutate}\NormalTok{(}
\NormalTok{    data6,}
    \DataTypeTok{FEBRE_aux1 =}\NormalTok{ FEBRE,}
    \DataTypeTok{TOSSE_aux1 =}\NormalTok{ TOSSE,}
    \DataTypeTok{GARGANTA_aux1 =}\NormalTok{ GARGANTA,}
    \DataTypeTok{DISPNEIA_aux1 =}\NormalTok{ DISPNEIA,}
    \DataTypeTok{DESC_RESP_aux1 =}\NormalTok{ DESC_RESP,}
    \DataTypeTok{SATURACAO_aux1 =}\NormalTok{ SATURACAO,}
    \DataTypeTok{DIARREIA_aux1 =}\NormalTok{ DIARREIA,}
    \DataTypeTok{VOMITO_aux1 =}\NormalTok{ VOMITO,}
    \DataTypeTok{DOR_ABD_aux1 =}\NormalTok{ DOR_ABD,}
    \DataTypeTok{FADIGA_aux1 =}\NormalTok{ FADIGA,}
    \DataTypeTok{PERD_OLFT_aux1 =}\NormalTok{ PERD_OLFT,}
    \DataTypeTok{PERD_PALA_aux1 =}\NormalTok{ PERD_PALA}
\NormalTok{  )}

\NormalTok{data6 <-}\StringTok{  }\NormalTok{data6 }\OperatorTok
\StringTok{  }\KeywordTok{mutate_at}\NormalTok{(}\KeywordTok{all_of}\NormalTok{(sintomas), }\ControlFlowTok{function}\NormalTok{(x) \{}
    \KeywordTok{case_when}\NormalTok{(x }\OperatorTok{==}\StringTok{ "1"} \OperatorTok{~}\StringTok{ }\DecValTok{1}\NormalTok{, }\OtherTok{TRUE} \OperatorTok{~}\StringTok{ }\DecValTok{0}\NormalTok{)}
\NormalTok{  \}) }\OperatorTok
\StringTok{  }\KeywordTok{mutate_at}\NormalTok{(}\KeywordTok{all_of}\NormalTok{(sintomas1), }\ControlFlowTok{function}\NormalTok{(x) \{}
    \KeywordTok{case_when}\NormalTok{(x }\OperatorTok{==}\StringTok{ "1"} \OperatorTok{~}\StringTok{ }\DecValTok{1}\NormalTok{, x }\OperatorTok{==}\StringTok{ "2"} \OperatorTok{~}\StringTok{ }\DecValTok{0}\NormalTok{, }\OtherTok{TRUE} \OperatorTok{~}\StringTok{ }\OtherTok{NA_real_}\NormalTok{)}
\NormalTok{  \}) }\OperatorTok
\StringTok{  }\KeywordTok{mutate}\NormalTok{(}
    \DataTypeTok{cont_sintomas =}\NormalTok{ FEBRE_aux }\OperatorTok{+}\StringTok{ }\NormalTok{TOSSE_aux }\OperatorTok{+}\StringTok{  }\NormalTok{GARGANTA_aux }\OperatorTok{+}\StringTok{ }\NormalTok{DISPNEIA_aux }\OperatorTok{+}\StringTok{ }\NormalTok{DESC_RESP_aux }\OperatorTok{+}
\StringTok{      }\NormalTok{SATURACAO_aux }\OperatorTok{+}\StringTok{ }\NormalTok{DIARREIA_aux }\OperatorTok{+}\StringTok{ }\NormalTok{VOMITO_aux }\OperatorTok{+}\StringTok{ }\NormalTok{DOR_ABD_aux }\OperatorTok{+}\StringTok{ }\NormalTok{FADIGA_aux }\OperatorTok{+}\StringTok{ }
\StringTok{     }\NormalTok{PERD_OLFT_aux }\OperatorTok{+}\StringTok{ }\NormalTok{PERD_PALA_aux}
\NormalTok{  ) }\OperatorTok
\StringTok{  }\KeywordTok{mutate}\NormalTok{(}
    \DataTypeTok{num_sintomas =} \KeywordTok{case_when}\NormalTok{(}
      \KeywordTok{is.na}\NormalTok{(FEBRE_aux1) }\OperatorTok{|}
\StringTok{        }\KeywordTok{is.na}\NormalTok{(TOSSE_aux1) }\OperatorTok{|}
\StringTok{        }\KeywordTok{is.na}\NormalTok{(GARGANTA_aux1) }\OperatorTok{|}
\StringTok{        }\KeywordTok{is.na}\NormalTok{(DISPNEIA_aux1) }\OperatorTok{|}
\StringTok{        }\KeywordTok{is.na}\NormalTok{(DESC_RESP_aux1) }\OperatorTok{|}
\StringTok{        }\KeywordTok{is.na}\NormalTok{(SATURACAO_aux1) }\OperatorTok{|}\StringTok{ }\KeywordTok{is.na}\NormalTok{(DIARREIA_aux1) }\OperatorTok{|}
\StringTok{        }\KeywordTok{is.na}\NormalTok{(VOMITO_aux1) }\OperatorTok{|}
\StringTok{        }\KeywordTok{is.na}\NormalTok{(DOR_ABD_aux1) }\OperatorTok{|}
\StringTok{        }\KeywordTok{is.na}\NormalTok{(FADIGA_aux1) }\OperatorTok{|}
\StringTok{        }\KeywordTok{is.na}\NormalTok{(PERD_OLFT_aux1) }\OperatorTok{|}\StringTok{ }\KeywordTok{is.na}\NormalTok{(PERD_PALA_aux1) }\OperatorTok{~}\StringTok{ }\OtherTok{NA_real_}\NormalTok{,}
      \OtherTok{TRUE} \OperatorTok{~}\StringTok{ }\NormalTok{cont_sintomas}
\NormalTok{    ),}
    \DataTypeTok{gr_sintomas =} \KeywordTok{case_when}\NormalTok{(}
\NormalTok{      num_sintomas }\OperatorTok{==}\StringTok{ }\DecValTok{0} \OperatorTok{~}\StringTok{ }\DecValTok{0}\NormalTok{,}
\NormalTok{      num_sintomas }\OperatorTok{==}\StringTok{ }\DecValTok{1} \OperatorTok{~}\StringTok{ }\DecValTok{1}\NormalTok{,}
\NormalTok{      num_sintomas }\OperatorTok{==}\StringTok{ }\DecValTok{2} \OperatorTok{~}\StringTok{ }\DecValTok{1}\NormalTok{,}
\NormalTok{      num_sintomas }\OperatorTok{>}\StringTok{  }\DecValTok{2} \OperatorTok{~}\StringTok{ }\DecValTok{2}\NormalTok{,}
      \OtherTok{TRUE} \OperatorTok{~}\StringTok{ }\OtherTok{NA_real_}
\NormalTok{    ),}
    \DataTypeTok{sintomas_SN =} \KeywordTok{case_when}\NormalTok{(}
\NormalTok{      gr_sintomas }\OperatorTok{==}\StringTok{ }\DecValTok{0} \OperatorTok{~}\StringTok{ }\DecValTok{0}\NormalTok{,}
\NormalTok{      gr_sintomas }\OperatorTok{==}\StringTok{ }\DecValTok{1} \OperatorTok{~}\StringTok{ }\DecValTok{1}\NormalTok{,}
\NormalTok{      gr_sintomas }\OperatorTok{==}\StringTok{ }\DecValTok{2} \OperatorTok{~}\StringTok{ }\DecValTok{1}\NormalTok{,}
      \OtherTok{TRUE} \OperatorTok{~}\StringTok{ }\OtherTok{NA_real_}
\NormalTok{    )}
\NormalTok{  )}

\CommentTok{#Symptom group }
\NormalTok{data6}\OperatorTok{$}\NormalTok{gr_sintomas <-}\StringTok{ }\KeywordTok{factor}\NormalTok{(}
\NormalTok{  data6}\OperatorTok{$}\NormalTok{gr_sintomas,}
  \DataTypeTok{levels =} \KeywordTok{c}\NormalTok{(}\DecValTok{0}\NormalTok{, }\DecValTok{1}\NormalTok{, }\DecValTok{2}\NormalTok{),}
  \DataTypeTok{labels =} \KeywordTok{c}\NormalTok{(}\StringTok{"none"}\NormalTok{, }\StringTok{"1 or 2"}\NormalTok{, }\StringTok{">2"}\NormalTok{)}
\NormalTok{)}

\CommentTok{#At least one symptom indicator }
\NormalTok{data6}\OperatorTok{$}\NormalTok{sintomas_SN <-}\StringTok{ }\KeywordTok{factor}\NormalTok{(data6}\OperatorTok{$}\NormalTok{sintomas_SN,}
                             \DataTypeTok{levels =} \KeywordTok{c}\NormalTok{(}\DecValTok{1}\NormalTok{, }\DecValTok{0}\NormalTok{),}
                             \DataTypeTok{labels =} \KeywordTok{c}\NormalTok{(}\StringTok{"yes"}\NormalTok{, }\StringTok{"no"}\NormalTok{))}
\end{Highlighting}
\end{Shaded}

\begin{Shaded}
\begin{Highlighting}[]
\NormalTok{questionr}\OperatorTok{::}\KeywordTok{freq}\NormalTok{(}
\NormalTok{  data6}\OperatorTok{$}\NormalTok{gr_sintomas,}
  \DataTypeTok{cum =} \OtherTok{FALSE}\NormalTok{,}
  \DataTypeTok{total =} \OtherTok{TRUE}\NormalTok{,}
  \DataTypeTok{na.last =} \OtherTok{FALSE}\NormalTok{,}
  \DataTypeTok{valid =} \OtherTok{FALSE}
\NormalTok{) }\OperatorTok
\StringTok{  }\NormalTok{knitr}\OperatorTok{::}\KeywordTok{kable}\NormalTok{(}\DataTypeTok{caption =} \StringTok{"Frequency table for symptom group"}\NormalTok{, }\DataTypeTok{digits =} \DecValTok{2}\NormalTok{) }\OperatorTok
\StringTok{  }\KeywordTok{kable_styling}\NormalTok{(}\DataTypeTok{latex_options =} \StringTok{"hold_position"}\NormalTok{)}
\end{Highlighting}
\end{Shaded}

\begin{table}[!h]

\caption{\label{tab:unnamed-chunk-74}Frequency table for symptom group}
\centering
\begin{tabular}[t]{l|r|r}
\hline
  & n & \%\\
\hline
none & 338 & 0.8\\
\hline
1 or 2 & 2347 & 5.8\\
\hline
>2 & 12028 & 29.6\\
\hline
NA & 25927 & 63.8\\
\hline
Total & 40640 & 100.0\\
\hline
\end{tabular}
\end{table}

\begin{Shaded}
\begin{Highlighting}[]
\NormalTok{questionr}\OperatorTok{::}\KeywordTok{freq}\NormalTok{(}
\NormalTok{  data6}\OperatorTok{$}\NormalTok{sintomas_SN,}
  \DataTypeTok{cum =} \OtherTok{FALSE}\NormalTok{,}
  \DataTypeTok{total =} \OtherTok{TRUE}\NormalTok{,}
  \DataTypeTok{na.last =} \OtherTok{FALSE}\NormalTok{,}
  \DataTypeTok{valid =} \OtherTok{FALSE}
\NormalTok{) }\OperatorTok
\StringTok{  }\NormalTok{knitr}\OperatorTok{::}\KeywordTok{kable}\NormalTok{(}\DataTypeTok{caption =} \StringTok{"Frequency table for at least one symptom indicator"}\NormalTok{, }\DataTypeTok{digits =} \DecValTok{2}\NormalTok{) }\OperatorTok
\StringTok{  }\KeywordTok{kable_styling}\NormalTok{(}\DataTypeTok{latex_options =} \StringTok{"hold_position"}\NormalTok{)}
\end{Highlighting}
\end{Shaded}

\begin{table}[!h]

\caption{\label{tab:unnamed-chunk-75}Frequency table for at least one symptom indicator}
\centering
\begin{tabular}[t]{l|r|r}
\hline
  & n & \%\\
\hline
yes & 14375 & 35.4\\
\hline
no & 338 & 0.8\\
\hline
NA & 25927 & 63.8\\
\hline
Total & 40640 & 100.0\\
\hline
\end{tabular}
\end{table}

An indicator variable of at least one respiratory symptom
(\texttt{sint\_resp}) is created in the following.

\begin{Shaded}
\begin{Highlighting}[]
\NormalTok{resp <-}\StringTok{  }\KeywordTok{c}\NormalTok{(}\StringTok{"DISPNEIA_aux"}\NormalTok{, }\StringTok{"DESC_RESP_aux"}\NormalTok{, }\StringTok{"SATURACAO_aux"}\NormalTok{)}

\NormalTok{resp1 <-}\StringTok{  }\KeywordTok{c}\NormalTok{(}\StringTok{"DISPNEIA_aux1"}\NormalTok{, }\StringTok{"DESC_RESP_aux1"}\NormalTok{, }\StringTok{"SATURACAO_aux1"}\NormalTok{)}

\NormalTok{data6 <-}
\StringTok{  }\KeywordTok{mutate}\NormalTok{(}
\NormalTok{    data6,}
    \DataTypeTok{DISPNEIA_aux =}\NormalTok{ DISPNEIA,}
    \DataTypeTok{DESC_RESP_aux =}\NormalTok{ DESC_RESP,}
    \DataTypeTok{SATURACAO_aux =}\NormalTok{ SATURACAO}
\NormalTok{  )}

\NormalTok{data6 <-}
\StringTok{  }\KeywordTok{mutate}\NormalTok{(}
\NormalTok{    data6,}
    \DataTypeTok{DISPNEIA_aux1 =}\NormalTok{ DISPNEIA,}
    \DataTypeTok{DESC_RESP_aux1 =}\NormalTok{ DESC_RESP,}
    \DataTypeTok{SATURACAO_aux1 =}\NormalTok{ SATURACAO}
\NormalTok{  )}

\NormalTok{data6 <-}\StringTok{  }\NormalTok{data6 }\OperatorTok
\StringTok{  }\KeywordTok{mutate_at}\NormalTok{(}\KeywordTok{all_of}\NormalTok{(resp), }\ControlFlowTok{function}\NormalTok{(x) \{}
    \KeywordTok{case_when}\NormalTok{(x }\OperatorTok{==}\StringTok{ "1"} \OperatorTok{~}\StringTok{ }\DecValTok{1}\NormalTok{, }\OtherTok{TRUE} \OperatorTok{~}\StringTok{ }\DecValTok{0}\NormalTok{)}
\NormalTok{  \}) }\OperatorTok
\StringTok{  }\KeywordTok{mutate_at}\NormalTok{(}\KeywordTok{all_of}\NormalTok{(resp1), }\ControlFlowTok{function}\NormalTok{(x) \{}
    \KeywordTok{case_when}\NormalTok{(x }\OperatorTok{==}\StringTok{ "1"} \OperatorTok{~}\StringTok{ }\DecValTok{1}\NormalTok{, x }\OperatorTok{==}\StringTok{ "2"} \OperatorTok{~}\StringTok{ }\DecValTok{0}\NormalTok{, }\OtherTok{TRUE} \OperatorTok{~}\StringTok{ }\OtherTok{NA_real_}\NormalTok{)}
\NormalTok{  \}) }\OperatorTok
\StringTok{  }\KeywordTok{mutate}\NormalTok{(}\DataTypeTok{cont_resp =}\NormalTok{ DISPNEIA_aux }\OperatorTok{+}\StringTok{ }\NormalTok{DESC_RESP_aux }\OperatorTok{+}\StringTok{ }\NormalTok{SATURACAO_aux) }\OperatorTok
\StringTok{  }\KeywordTok{mutate}\NormalTok{(}
    \DataTypeTok{num_resp =} \KeywordTok{case_when}\NormalTok{(}
\NormalTok{      (cont_resp }\OperatorTok{==}\StringTok{ }\DecValTok{0}\NormalTok{) }\OperatorTok{&}
\StringTok{        }\NormalTok{(}
          \KeywordTok{is.na}\NormalTok{(DISPNEIA_aux1) }\OperatorTok{|}
\StringTok{            }\KeywordTok{is.na}\NormalTok{(DESC_RESP_aux1) }\OperatorTok{|}\StringTok{ }\KeywordTok{is.na}\NormalTok{(SATURACAO_aux1)}
\NormalTok{        ) }\OperatorTok{~}\StringTok{ }\OtherTok{NA_real_}\NormalTok{,}
      \OtherTok{TRUE} \OperatorTok{~}\StringTok{ }\NormalTok{cont_resp}
\NormalTok{    ),}
    \DataTypeTok{sint_resp =} \KeywordTok{case_when}\NormalTok{(}
\NormalTok{      num_resp }\OperatorTok{==}\StringTok{ }\DecValTok{0} \OperatorTok{~}\StringTok{ }\DecValTok{0}\NormalTok{,}
\NormalTok{      num_resp }\OperatorTok{==}\StringTok{ }\DecValTok{1} \OperatorTok{~}\StringTok{ }\DecValTok{1}\NormalTok{,}
\NormalTok{      num_resp }\OperatorTok{==}\StringTok{ }\DecValTok{2} \OperatorTok{~}\StringTok{ }\DecValTok{1}\NormalTok{,}
\NormalTok{      num_resp }\OperatorTok{==}\StringTok{ }\DecValTok{3} \OperatorTok{~}\StringTok{ }\DecValTok{1}\NormalTok{,}
      \OtherTok{TRUE} \OperatorTok{~}\StringTok{ }\OtherTok{NA_real_}
\NormalTok{    )}
\NormalTok{  )}

\CommentTok{# Any respiratory symptom indicator}
\NormalTok{data6}\OperatorTok{$}\NormalTok{sint_resp <-}\StringTok{ }\KeywordTok{factor}\NormalTok{(data6}\OperatorTok{$}\NormalTok{sint_resp,}
                          \DataTypeTok{levels =} \KeywordTok{c}\NormalTok{(}\DecValTok{1}\NormalTok{, }\DecValTok{0}\NormalTok{),}
                          \DataTypeTok{labels =} \KeywordTok{c}\NormalTok{(}\StringTok{"yes"}\NormalTok{, }\StringTok{"no"}\NormalTok{))}
\end{Highlighting}
\end{Shaded}

\begin{Shaded}
\begin{Highlighting}[]
\NormalTok{questionr}\OperatorTok{::}\KeywordTok{freq}\NormalTok{(}
\NormalTok{  data6}\OperatorTok{$}\NormalTok{sint_resp,}
  \DataTypeTok{cum =} \OtherTok{FALSE}\NormalTok{,}
  \DataTypeTok{total =} \OtherTok{TRUE}\NormalTok{,}
  \DataTypeTok{na.last =} \OtherTok{FALSE}\NormalTok{,}
  \DataTypeTok{valid =} \OtherTok{FALSE}
\NormalTok{) }\OperatorTok
\StringTok{  }\NormalTok{knitr}\OperatorTok{::}\KeywordTok{kable}\NormalTok{(}\DataTypeTok{caption =} \StringTok{"Frequency table for any respiratory symptom"}\NormalTok{, }\DataTypeTok{digits =} \DecValTok{2}\NormalTok{) }\OperatorTok
\StringTok{  }\KeywordTok{kable_styling}\NormalTok{(}\DataTypeTok{latex_options =} \StringTok{"hold_position"}\NormalTok{)}
\end{Highlighting}
\end{Shaded}

\begin{table}[!h]

\caption{\label{tab:unnamed-chunk-77}Frequency table for any respiratory symptom}
\centering
\begin{tabular}[t]{l|r|r}
\hline
  & n & \%\\
\hline
yes & 32875 & 80.9\\
\hline
no & 4851 & 11.9\\
\hline
NA & 2914 & 7.2\\
\hline
Total & 40640 & 100.0\\
\hline
\end{tabular}
\end{table}

The SARI (severe acute respiratory infection) indicator (\texttt{sari})
is ``yes'' if one has fever and cough or sore throat and respiratory
distress or dyspnoea or saturation. The SARI without fever indicator
(\texttt{sari\_sfebre}) is what the name says.

\begin{Shaded}
\begin{Highlighting}[]
\NormalTok{data6 <-}\StringTok{  }\NormalTok{data6 }\OperatorTok
\StringTok{  }\KeywordTok{mutate}\NormalTok{(}
    \DataTypeTok{sari =} \KeywordTok{case_when}\NormalTok{(}
\NormalTok{      FEBRE }\OperatorTok{==}\StringTok{ "1"} \OperatorTok{&}
\StringTok{        }\NormalTok{(TOSSE }\OperatorTok{==}\StringTok{ "1"} \OperatorTok{|}\StringTok{ }\NormalTok{GARGANTA }\OperatorTok{==}\StringTok{ "1"}\NormalTok{) }\OperatorTok{&}
\StringTok{        }\NormalTok{(DESC_RESP }\OperatorTok{==}\StringTok{ "1"} \OperatorTok{|}
\StringTok{           }\NormalTok{DISPNEIA }\OperatorTok{==}\StringTok{ "1"} \OperatorTok{|}\StringTok{ }\NormalTok{SATURACAO }\OperatorTok{==}\StringTok{ "1"}\NormalTok{) }\OperatorTok{~}\StringTok{ }\DecValTok{1}\NormalTok{,}
      \KeywordTok{is.na}\NormalTok{(FEBRE_aux1) }\OperatorTok{|}
\StringTok{        }\NormalTok{(}\KeywordTok{is.na}\NormalTok{(TOSSE_aux1)  }\OperatorTok{&}
\StringTok{           }\KeywordTok{is.na}\NormalTok{(GARGANTA_aux1)) }\OperatorTok{|}
\StringTok{        }\NormalTok{(}
          \KeywordTok{is.na}\NormalTok{(DESC_RESP_aux1) }\OperatorTok{&}
\StringTok{            }\KeywordTok{is.na}\NormalTok{(DISPNEIA_aux1) }\OperatorTok{&}\StringTok{  }\KeywordTok{is.na}\NormalTok{(SATURACAO_aux1)}
\NormalTok{        ) }\OperatorTok{~}\StringTok{ }\OtherTok{NA_real_}\NormalTok{,}
      \OtherTok{TRUE} \OperatorTok{~}\StringTok{ }\DecValTok{0}
\NormalTok{    ),}
    \DataTypeTok{sari_sfebre =} \KeywordTok{case_when}\NormalTok{(}
\NormalTok{      (TOSSE }\OperatorTok{==}\StringTok{ "1"} \OperatorTok{|}\StringTok{ }\NormalTok{GARGANTA }\OperatorTok{==}\StringTok{ "1"}\NormalTok{) }\OperatorTok{&}
\StringTok{        }\NormalTok{(DESC_RESP }\OperatorTok{==}\StringTok{ "1"} \OperatorTok{|}\StringTok{ }\NormalTok{DISPNEIA }\OperatorTok{==}\StringTok{ "1"} \OperatorTok{|}\StringTok{ }\NormalTok{SATURACAO }\OperatorTok{==}\StringTok{ "1"}\NormalTok{) }\OperatorTok{~}\StringTok{ }\DecValTok{1}\NormalTok{,}
\NormalTok{      (}\KeywordTok{is.na}\NormalTok{(TOSSE_aux1)  }\OperatorTok{&}
\StringTok{         }\KeywordTok{is.na}\NormalTok{(GARGANTA_aux1)) }\OperatorTok{|}
\StringTok{        }\NormalTok{(}
          \KeywordTok{is.na}\NormalTok{(DESC_RESP_aux1) }\OperatorTok{&}
\StringTok{            }\KeywordTok{is.na}\NormalTok{(DISPNEIA_aux1) }\OperatorTok{&}\StringTok{  }\KeywordTok{is.na}\NormalTok{(SATURACAO_aux1)}
\NormalTok{        ) }\OperatorTok{~}\StringTok{ }\OtherTok{NA_real_}\NormalTok{,}
      \OtherTok{TRUE} \OperatorTok{~}\StringTok{ }\DecValTok{0}
\NormalTok{    )}
\NormalTok{  )}

\CommentTok{#SARI}
\NormalTok{data6}\OperatorTok{$}\NormalTok{sari <-}\StringTok{ }\KeywordTok{factor}\NormalTok{(data6}\OperatorTok{$}\NormalTok{sari,}
                      \DataTypeTok{levels =} \KeywordTok{c}\NormalTok{(}\DecValTok{1}\NormalTok{, }\DecValTok{0}\NormalTok{),}
                      \DataTypeTok{labels =} \KeywordTok{c}\NormalTok{(}\StringTok{"yes"}\NormalTok{, }\StringTok{"no"}\NormalTok{))}
\CommentTok{#SARI without fever}
\NormalTok{data6}\OperatorTok{$}\NormalTok{sari_sfebre <-}\StringTok{ }\KeywordTok{factor}\NormalTok{(data6}\OperatorTok{$}\NormalTok{sari_sfebre,}
                             \DataTypeTok{levels =} \KeywordTok{c}\NormalTok{(}\DecValTok{1}\NormalTok{, }\DecValTok{0}\NormalTok{),}
                             \DataTypeTok{labels =} \KeywordTok{c}\NormalTok{(}\StringTok{"yes"}\NormalTok{, }\StringTok{"no"}\NormalTok{))}
\end{Highlighting}
\end{Shaded}

\begin{Shaded}
\begin{Highlighting}[]
\NormalTok{questionr}\OperatorTok{::}\KeywordTok{freq}\NormalTok{(}
\NormalTok{  data6}\OperatorTok{$}\NormalTok{sari,}
  \DataTypeTok{cum =} \OtherTok{FALSE}\NormalTok{,}
  \DataTypeTok{total =} \OtherTok{TRUE}\NormalTok{,}
  \DataTypeTok{na.last =} \OtherTok{FALSE}\NormalTok{,}
  \DataTypeTok{valid =} \OtherTok{FALSE}
\NormalTok{) }\OperatorTok
\StringTok{  }\NormalTok{knitr}\OperatorTok{::}\KeywordTok{kable}\NormalTok{(}\DataTypeTok{caption =} \StringTok{"Frequency table for SARI"}\NormalTok{, }\DataTypeTok{digits =} \DecValTok{2}\NormalTok{) }\OperatorTok
\StringTok{  }\KeywordTok{kable_styling}\NormalTok{(}\DataTypeTok{latex_options =} \StringTok{"hold_position"}\NormalTok{)}
\end{Highlighting}
\end{Shaded}

\begin{table}[!h]

\caption{\label{tab:unnamed-chunk-79}Frequency table for SARI}
\centering
\begin{tabular}[t]{l|r|r}
\hline
  & n & \%\\
\hline
yes & 19639 & 48.3\\
\hline
no & 14479 & 35.6\\
\hline
NA & 6522 & 16.0\\
\hline
Total & 40640 & 100.0\\
\hline
\end{tabular}
\end{table}

\begin{Shaded}
\begin{Highlighting}[]
\NormalTok{questionr}\OperatorTok{::}\KeywordTok{freq}\NormalTok{(}
\NormalTok{  data6}\OperatorTok{$}\NormalTok{sari_sfebre,}
  \DataTypeTok{cum =} \OtherTok{FALSE}\NormalTok{,}
  \DataTypeTok{total =} \OtherTok{TRUE}\NormalTok{,}
  \DataTypeTok{na.last =} \OtherTok{FALSE}\NormalTok{,}
  \DataTypeTok{valid =} \OtherTok{FALSE}
\NormalTok{) }\OperatorTok
\StringTok{  }\NormalTok{knitr}\OperatorTok{::}\KeywordTok{kable}\NormalTok{(}\DataTypeTok{caption =} \StringTok{"Frequency table for SARI without fever"}\NormalTok{, }\DataTypeTok{digits =} \DecValTok{2}\NormalTok{) }\OperatorTok
\StringTok{  }\KeywordTok{kable_styling}\NormalTok{(}\DataTypeTok{latex_options =} \StringTok{"hold_position"}\NormalTok{)}
\end{Highlighting}
\end{Shaded}

\begin{table}[!h]

\caption{\label{tab:unnamed-chunk-80}Frequency table for SARI without fever}
\centering
\begin{tabular}[t]{l|r|r}
\hline
  & n & \%\\
\hline
yes & 26671 & 65.6\\
\hline
no & 9268 & 22.8\\
\hline
NA & 4701 & 11.6\\
\hline
Total & 40640 & 100.0\\
\hline
\end{tabular}
\end{table}

\hypertarget{propensity-score-matching-psm}{%
\section{4. Propensity score matching
(PSM)}\label{propensity-score-matching-psm}}

\hypertarget{psm-for-symptoms-variables}{%
\subsection{4.1 PSM for symptoms
variables}\label{psm-for-symptoms-variables}}

We considered as control variables: age, ethnicity, cardiopathy, asthma,
diabetes, immunodepression and obesity.

First, we present the difference result among the groups regarding the
control variables before the PSM. We consider as ``balanced'' the cases
with mean difference greater than 0.05. As we can see, for most
categories of control variables the groups are not balanced before the
matching.

\begin{Shaded}
\begin{Highlighting}[]
\KeywordTok{bal.tab}\NormalTok{(gest_puerp }\OperatorTok{~}\StringTok{ }\NormalTok{NU_IDADE_N }\OperatorTok{+}\StringTok{ }\NormalTok{raca }\OperatorTok{+}\StringTok{ }\NormalTok{cardiopati }\OperatorTok{+}\StringTok{ }\NormalTok{asma }\OperatorTok{+}\StringTok{ }\NormalTok{diabetes }\OperatorTok{+}\StringTok{ }\NormalTok{imunodepre }\OperatorTok{+}\StringTok{ }\NormalTok{obesidade, }\DataTypeTok{data =}\NormalTok{ data6, }\DataTypeTok{estimand =} \StringTok{"ATT"}\NormalTok{, }\DataTypeTok{m.threshold =} \FloatTok{.05}\NormalTok{)}
\end{Highlighting}
\end{Shaded}

\begin{verbatim}
## Balance summary across all treatment pairs
##                    Type Max.Diff.Un      M.Threshold.Un
## NU_IDADE_N_     Contin.      1.1385 Not Balanced, >0.05
## raca_white       Binary      0.1686 Not Balanced, >0.05
## raca_black       Binary      0.0130     Balanced, <0.05
## raca_yellow      Binary      0.0063     Balanced, <0.05
## raca_brown       Binary      0.1590 Not Balanced, >0.05
## raca_indigenous  Binary      0.0028     Balanced, <0.05
## raca:<NA>        Binary      0.0251     Balanced, <0.05
## cardiopati_no    Binary      0.2811 Not Balanced, >0.05
## cardiopati:<NA>  Binary      0.2976 Not Balanced, >0.05
## asma_no          Binary      0.0914 Not Balanced, >0.05
## asma:<NA>        Binary      0.2880 Not Balanced, >0.05
## diabetes_no      Binary      0.2486 Not Balanced, >0.05
## diabetes:<NA>    Binary      0.2758 Not Balanced, >0.05
## imunodepre_no    Binary      0.0742 Not Balanced, >0.05
## imunodepre:<NA>  Binary      0.2884 Not Balanced, >0.05
## obesidade_no     Binary      0.1981 Not Balanced, >0.05
## obesidade:<NA>   Binary      0.2805 Not Balanced, >0.05
## 
## Balance tally for mean differences
##                     count
## Balanced, <0.05         4
## Not Balanced, >0.05    13
## 
## Variable with the greatest mean difference
##     Variable Max.Diff.Un      M.Threshold.Un
##  NU_IDADE_N_      1.1385 Not Balanced, >0.05
## 
## Sample sizes
##        no preg puerp
## All 36474 3372   794
\end{verbatim}

After PSM, all categories of control variables are balanced, with the
exception of age (table below). Although the mean difference is greater
than 0.05, it substantially decreased after the PSM: from 1.1385 to
0.2684.

\begin{Shaded}
\begin{Highlighting}[]
\NormalTok{w.out <-}
\StringTok{  }\KeywordTok{weightit}\NormalTok{(}
\NormalTok{    gest_puerp }\OperatorTok{~}\StringTok{ }\NormalTok{NU_IDADE_N }\OperatorTok{+}\StringTok{ }\NormalTok{raca }\OperatorTok{+}\StringTok{ }\NormalTok{cardiopati }\OperatorTok{+}\StringTok{ }\NormalTok{asma }\OperatorTok{+}\StringTok{ }\NormalTok{diabetes }\OperatorTok{+}\StringTok{ }\NormalTok{imunodepre }\OperatorTok{+}\StringTok{ }\NormalTok{obesidade, }
    \DataTypeTok{use.mlogit =} \OtherTok{FALSE}\NormalTok{,}
    \DataTypeTok{data =}\NormalTok{ data6,}
    \DataTypeTok{focal =} \StringTok{"puerp"}\NormalTok{,}
    \DataTypeTok{method =} \StringTok{"ps"}\NormalTok{,}
    \DataTypeTok{estimand =} \StringTok{"ATT"}
\NormalTok{  ) }
\end{Highlighting}
\end{Shaded}

\begin{Shaded}
\begin{Highlighting}[]
\NormalTok{cobalt}\OperatorTok{::}\KeywordTok{bal.tab}\NormalTok{(w.out, }\DataTypeTok{m.threshold =} \FloatTok{0.05}\NormalTok{, }\DataTypeTok{disp.v.ratio =} \OtherTok{TRUE}\NormalTok{)}
\end{Highlighting}
\end{Shaded}

\begin{verbatim}
## Call
##  weightit(formula = gest_puerp ~ NU_IDADE_N + raca + cardiopati + 
##     asma + diabetes + imunodepre + obesidade, data = data6, method = "ps", 
##     estimand = "ATT", focal = "puerp", use.mlogit = FALSE)
## 
## Balance summary across all treatment pairs
##                    Type Max.Diff.Adj         M.Threshold Max.V.Ratio.Adj
## NU_IDADE_N      Contin.       0.2684 Not Balanced, >0.05          2.2665
## raca_white       Binary       0.0080     Balanced, <0.05               .
## raca_black       Binary       0.0035     Balanced, <0.05               .
## raca_yellow      Binary       0.0011     Balanced, <0.05               .
## raca_brown       Binary       0.0105     Balanced, <0.05               .
## raca_indigenous  Binary       0.0008     Balanced, <0.05               .
## raca:<NA>        Binary       0.0062     Balanced, <0.05               .
## cardiopati_no    Binary       0.0104     Balanced, <0.05               .
## cardiopati:<NA>  Binary       0.0099     Balanced, <0.05               .
## asma_no          Binary       0.0045     Balanced, <0.05               .
## asma:<NA>        Binary       0.0122     Balanced, <0.05               .
## diabetes_no      Binary       0.0077     Balanced, <0.05               .
## diabetes:<NA>    Binary       0.0120     Balanced, <0.05               .
## imunodepre_no    Binary       0.0019     Balanced, <0.05               .
## imunodepre:<NA>  Binary       0.0120     Balanced, <0.05               .
## obesidade_no     Binary       0.0095     Balanced, <0.05               .
## obesidade:<NA>   Binary       0.0122     Balanced, <0.05               .
## 
## Balance tally for mean differences
##                     count
## Balanced, <0.05        16
## Not Balanced, >0.05     1
## 
## Variable with the greatest mean difference
##    Variable Max.Diff.Adj         M.Threshold
##  NU_IDADE_N       0.2684 Not Balanced, >0.05
## 
## Effective sample sizes
##                  no    preg puerp
## Unadjusted 36474.   3372.     794
## Adjusted    3476.39 2296.34   794
\end{verbatim}

\hypertarget{psm-for-outcomes}{%
\subsection{4.2 PSM for outcomes}\label{psm-for-outcomes}}

For the analysis of the outcomes, only the cases that we know whether it
is a case of death or cure are selected.

\begin{Shaded}
\begin{Highlighting}[]
\NormalTok{data6 <-}
\StringTok{  }\NormalTok{data6 }\OperatorTok\StringTok{ }\KeywordTok{mutate}\NormalTok{(}
    \DataTypeTok{evolucao =} \KeywordTok{case_when}\NormalTok{(}
\NormalTok{      EVOLUCAO }\OperatorTok{==}\StringTok{ }\DecValTok{1} \OperatorTok{~}\StringTok{ "cure"}\NormalTok{,}
\NormalTok{      EVOLUCAO }\OperatorTok{==}\StringTok{ }\DecValTok{2} \OperatorTok{~}\StringTok{ "death"}\NormalTok{,}
\NormalTok{      EVOLUCAO }\OperatorTok{==}\StringTok{ }\DecValTok{3} \OperatorTok{~}\StringTok{ "death"}\NormalTok{,}
      \OtherTok{TRUE} \OperatorTok{~}\StringTok{ "in progress"}
\NormalTok{    )}
\NormalTok{  )}
\end{Highlighting}
\end{Shaded}

Now we exclude cases ``in progress''.

\begin{Shaded}
\begin{Highlighting}[]
\NormalTok{data7 <-}\StringTok{ }\KeywordTok{filter}\NormalTok{(data6, evolucao }\OperatorTok{!=}\StringTok{ "in progress"}\NormalTok{)}

\NormalTok{data7}\OperatorTok{$}\NormalTok{evolucao <-}\StringTok{ }\KeywordTok{factor}\NormalTok{(}
\NormalTok{  data7}\OperatorTok{$}\NormalTok{evolucao,}
  \DataTypeTok{levels =} \KeywordTok{c}\NormalTok{(}\StringTok{"death"}\NormalTok{, }\StringTok{"cure"}\NormalTok{),}
  \DataTypeTok{labels =} \KeywordTok{c}\NormalTok{(}\StringTok{"death"}\NormalTok{, }\StringTok{"cure"}\NormalTok{)}
\NormalTok{)}
\end{Highlighting}
\end{Shaded}

In this propensity score, we considered as control variables: age,
ethnicity, school, Brazilian Federative Unit, cardiopathy, asthma,
diabetes, immunodepression, obesity and respiratory symptoms.

We present below the difference result among the groups regarding the
control variables before the PSM. We consider as ``balanced'' the cases
with mean difference greater than 0.05. As we can see, the groups are
not balanced before the matching for many categories of the control
variables.

\begin{Shaded}
\begin{Highlighting}[]
\KeywordTok{bal.tab}\NormalTok{(gest_puerp }\OperatorTok{~}\StringTok{ }\NormalTok{NU_IDADE_N }\OperatorTok{+}\StringTok{ }\NormalTok{raca }\OperatorTok{+}\StringTok{ }\NormalTok{escol }\OperatorTok{+}\StringTok{ }\NormalTok{SG_UF }\OperatorTok{+}\StringTok{ }\NormalTok{cardiopati }\OperatorTok{+}\StringTok{ }\NormalTok{asma }\OperatorTok{+}\StringTok{ }
\StringTok{                    }\NormalTok{diabetes }\OperatorTok{+}\StringTok{ }\NormalTok{imunodepre }\OperatorTok{+}\StringTok{ }\NormalTok{obesidade }\OperatorTok{+}\StringTok{ }\NormalTok{sint_resp, }\DataTypeTok{data =}\NormalTok{ data7, }\DataTypeTok{estimand =} \StringTok{"ATT"}\NormalTok{, }\DataTypeTok{m.threshold =} \FloatTok{0.05}\NormalTok{)}
\end{Highlighting}
\end{Shaded}

\begin{verbatim}
## Balance summary across all treatment pairs
##                            Type Max.Diff.Un      M.Threshold.Un
## NU_IDADE_N_             Contin.      1.1475 Not Balanced, >0.05
## raca_white               Binary      0.1723 Not Balanced, >0.05
## raca_black               Binary      0.0126     Balanced, <0.05
## raca_yellow              Binary      0.0070     Balanced, <0.05
## raca_brown               Binary      0.1658 Not Balanced, >0.05
## raca_indigenous          Binary      0.0036     Balanced, <0.05
## raca:<NA>                Binary      0.0330     Balanced, <0.05
## escol_no education       Binary      0.0098     Balanced, <0.05
## escol_up to high school  Binary      0.0089     Balanced, <0.05
## escol_high school        Binary      0.0720 Not Balanced, >0.05
## escol_higher education   Binary      0.0711 Not Balanced, >0.05
## escol:<NA>               Binary      0.0077     Balanced, <0.05
## SG_UF_AC                 Binary      0.0037     Balanced, <0.05
## SG_UF_AL                 Binary      0.0025     Balanced, <0.05
## SG_UF_AM                 Binary      0.0222     Balanced, <0.05
## SG_UF_AP                 Binary      0.0007     Balanced, <0.05
## SG_UF_BA                 Binary      0.0293     Balanced, <0.05
## SG_UF_CE                 Binary      0.0349     Balanced, <0.05
## SG_UF_DF                 Binary      0.0315     Balanced, <0.05
## SG_UF_ES                 Binary      0.0036     Balanced, <0.05
## SG_UF_GO                 Binary      0.0161     Balanced, <0.05
## SG_UF_MA                 Binary      0.0130     Balanced, <0.05
## SG_UF_MG                 Binary      0.0237     Balanced, <0.05
## SG_UF_MS                 Binary      0.0132     Balanced, <0.05
## SG_UF_MT                 Binary      0.0086     Balanced, <0.05
## SG_UF_PA                 Binary      0.0128     Balanced, <0.05
## SG_UF_PB                 Binary      0.0105     Balanced, <0.05
## SG_UF_PE                 Binary      0.0631 Not Balanced, >0.05
## SG_UF_PI                 Binary      0.0137     Balanced, <0.05
## SG_UF_PR                 Binary      0.0198     Balanced, <0.05
## SG_UF_RJ                 Binary      0.0178     Balanced, <0.05
## SG_UF_RN                 Binary      0.0095     Balanced, <0.05
## SG_UF_RO                 Binary      0.0052     Balanced, <0.05
## SG_UF_RR                 Binary      0.0009     Balanced, <0.05
## SG_UF_RS                 Binary      0.0281     Balanced, <0.05
## SG_UF_SC                 Binary      0.0090     Balanced, <0.05
## SG_UF_SE                 Binary      0.0088     Balanced, <0.05
## SG_UF_SP                 Binary      0.1169 Not Balanced, >0.05
## SG_UF_TO                 Binary      0.0021     Balanced, <0.05
## SG_UF:<NA>               Binary      0.0014     Balanced, <0.05
## cardiopati_no            Binary      0.2775 Not Balanced, >0.05
## cardiopati:<NA>          Binary      0.3056 Not Balanced, >0.05
## asma_no                  Binary      0.0954 Not Balanced, >0.05
## asma:<NA>                Binary      0.2868 Not Balanced, >0.05
## diabetes_no              Binary      0.2381 Not Balanced, >0.05
## diabetes:<NA>            Binary      0.2843 Not Balanced, >0.05
## imunodepre_no            Binary      0.0728 Not Balanced, >0.05
## imunodepre:<NA>          Binary      0.2930 Not Balanced, >0.05
## obesidade_no             Binary      0.1970 Not Balanced, >0.05
## obesidade:<NA>           Binary      0.2858 Not Balanced, >0.05
## sint_resp_no             Binary      0.1953 Not Balanced, >0.05
## sint_resp:<NA>           Binary      0.0485     Balanced, <0.05
## 
## Balance tally for mean differences
##                     count
## Balanced, <0.05        34
## Not Balanced, >0.05    18
## 
## Variable with the greatest mean difference
##     Variable Max.Diff.Un      M.Threshold.Un
##  NU_IDADE_N_      1.1475 Not Balanced, >0.05
## 
## Sample sizes
##        no preg puerp
## All 32081 2904   715
\end{verbatim}

After PSM, all categories of control variables are balanced, with the
exception of age and the ``up to high school'' category (table below).
Despite this, we can see that the mean difference value for the ``up to
high school'' category (value of 0.0502) is very close to the threshold
and for age, the mean difference is substantially decreased after the
PSM: from 1.1475 to 0.3693.

\begin{Shaded}
\begin{Highlighting}[]
\KeywordTok{remove}\NormalTok{(w.out)}

\NormalTok{w.out <-}\StringTok{ }\KeywordTok{weightit}\NormalTok{(gest_puerp }\OperatorTok{~}\StringTok{ }\NormalTok{NU_IDADE_N }\OperatorTok{+}\StringTok{ }\NormalTok{raca }\OperatorTok{+}\StringTok{ }\NormalTok{escol }\OperatorTok{+}\StringTok{ }\NormalTok{SG_UF }\OperatorTok{+}\StringTok{ }\NormalTok{cardiopati }\OperatorTok{+}\StringTok{ }\NormalTok{asma }\OperatorTok{+}\StringTok{ }
\StringTok{                    }\NormalTok{diabetes }\OperatorTok{+}\StringTok{ }\NormalTok{imunodepre }\OperatorTok{+}\StringTok{ }\NormalTok{obesidade }\OperatorTok{+}\StringTok{ }\NormalTok{sint_resp, }\DataTypeTok{use.mlogit =} \OtherTok{FALSE}\NormalTok{, }
                  \DataTypeTok{data =}\NormalTok{ data7, }\DataTypeTok{focal =} \StringTok{"puerp"}\NormalTok{, }\DataTypeTok{method =} \StringTok{"ps"}\NormalTok{, }\DataTypeTok{estimand =} \StringTok{"ATT"}\NormalTok{)}
\end{Highlighting}
\end{Shaded}

\begin{Shaded}
\begin{Highlighting}[]
\NormalTok{cobalt}\OperatorTok{::}\KeywordTok{bal.tab}\NormalTok{(w.out, }\DataTypeTok{m.threshold =} \FloatTok{0.05}\NormalTok{, }\DataTypeTok{disp.v.ratio =} \OtherTok{TRUE}\NormalTok{)}
\end{Highlighting}
\end{Shaded}

\begin{verbatim}
## Call
##  weightit(formula = gest_puerp ~ NU_IDADE_N + raca + escol + SG_UF + 
##     cardiopati + asma + diabetes + imunodepre + obesidade + sint_resp, 
##     data = data7, method = "ps", estimand = "ATT", focal = "puerp", 
##     use.mlogit = FALSE)
## 
## Balance summary across all treatment pairs
##                            Type Max.Diff.Adj         M.Threshold
## NU_IDADE_N              Contin.       0.3693 Not Balanced, >0.05
## raca_white               Binary       0.0108     Balanced, <0.05
## raca_black               Binary       0.0077     Balanced, <0.05
## raca_yellow              Binary       0.0025     Balanced, <0.05
## raca_brown               Binary       0.0207     Balanced, <0.05
## raca_indigenous          Binary       0.0014     Balanced, <0.05
## raca:<NA>                Binary       0.0064     Balanced, <0.05
## escol_no education       Binary       0.0041     Balanced, <0.05
## escol_up to high school  Binary       0.0502 Not Balanced, >0.05
## escol_high school        Binary       0.0450     Balanced, <0.05
## escol_higher education   Binary       0.0066     Balanced, <0.05
## escol:<NA>               Binary       0.0342     Balanced, <0.05
## SG_UF_AC                 Binary       0.0002     Balanced, <0.05
## SG_UF_AL                 Binary       0.0011     Balanced, <0.05
## SG_UF_AM                 Binary       0.0054     Balanced, <0.05
## SG_UF_AP                 Binary       0.0003     Balanced, <0.05
## SG_UF_BA                 Binary       0.0007     Balanced, <0.05
## SG_UF_CE                 Binary       0.0010     Balanced, <0.05
## SG_UF_DF                 Binary       0.0119     Balanced, <0.05
## SG_UF_ES                 Binary       0.0032     Balanced, <0.05
## SG_UF_GO                 Binary       0.0043     Balanced, <0.05
## SG_UF_MA                 Binary       0.0010     Balanced, <0.05
## SG_UF_MG                 Binary       0.0082     Balanced, <0.05
## SG_UF_MS                 Binary       0.0020     Balanced, <0.05
## SG_UF_MT                 Binary       0.0037     Balanced, <0.05
## SG_UF_PA                 Binary       0.0008     Balanced, <0.05
## SG_UF_PB                 Binary       0.0048     Balanced, <0.05
## SG_UF_PE                 Binary       0.0328     Balanced, <0.05
## SG_UF_PI                 Binary       0.0002     Balanced, <0.05
## SG_UF_PR                 Binary       0.0045     Balanced, <0.05
## SG_UF_RJ                 Binary       0.0051     Balanced, <0.05
## SG_UF_RN                 Binary       0.0031     Balanced, <0.05
## SG_UF_RO                 Binary       0.0008     Balanced, <0.05
## SG_UF_RR                 Binary       0.0002     Balanced, <0.05
## SG_UF_RS                 Binary       0.0058     Balanced, <0.05
## SG_UF_SC                 Binary       0.0038     Balanced, <0.05
## SG_UF_SE                 Binary       0.0029     Balanced, <0.05
## SG_UF_SP                 Binary       0.0129     Balanced, <0.05
## SG_UF_TO                 Binary       0.0008     Balanced, <0.05
## SG_UF:<NA>               Binary       0.0014     Balanced, <0.05
## cardiopati_no            Binary       0.0149     Balanced, <0.05
## cardiopati:<NA>          Binary       0.0101     Balanced, <0.05
## asma_no                  Binary       0.0060     Balanced, <0.05
## asma:<NA>                Binary       0.0138     Balanced, <0.05
## diabetes_no              Binary       0.0084     Balanced, <0.05
## diabetes:<NA>            Binary       0.0128     Balanced, <0.05
## imunodepre_no            Binary       0.0026     Balanced, <0.05
## imunodepre:<NA>          Binary       0.0116     Balanced, <0.05
## obesidade_no             Binary       0.0095     Balanced, <0.05
## obesidade:<NA>           Binary       0.0138     Balanced, <0.05
## sint_resp_no             Binary       0.0061     Balanced, <0.05
## sint_resp:<NA>           Binary       0.0159     Balanced, <0.05
##                         Max.V.Ratio.Adj
## NU_IDADE_N                       2.4696
## raca_white                            .
## raca_black                            .
## raca_yellow                           .
## raca_brown                            .
## raca_indigenous                       .
## raca:<NA>                             .
## escol_no education                    .
## escol_up to high school               .
## escol_high school                     .
## escol_higher education                .
## escol:<NA>                            .
## SG_UF_AC                              .
## SG_UF_AL                              .
## SG_UF_AM                              .
## SG_UF_AP                              .
## SG_UF_BA                              .
## SG_UF_CE                              .
## SG_UF_DF                              .
## SG_UF_ES                              .
## SG_UF_GO                              .
## SG_UF_MA                              .
## SG_UF_MG                              .
## SG_UF_MS                              .
## SG_UF_MT                              .
## SG_UF_PA                              .
## SG_UF_PB                              .
## SG_UF_PE                              .
## SG_UF_PI                              .
## SG_UF_PR                              .
## SG_UF_RJ                              .
## SG_UF_RN                              .
## SG_UF_RO                              .
## SG_UF_RR                              .
## SG_UF_RS                              .
## SG_UF_SC                              .
## SG_UF_SE                              .
## SG_UF_SP                              .
## SG_UF_TO                              .
## SG_UF:<NA>                            .
## cardiopati_no                         .
## cardiopati:<NA>                       .
## asma_no                               .
## asma:<NA>                             .
## diabetes_no                           .
## diabetes:<NA>                         .
## imunodepre_no                         .
## imunodepre:<NA>                       .
## obesidade_no                          .
## obesidade:<NA>                        .
## sint_resp_no                          .
## sint_resp:<NA>                        .
## 
## Balance tally for mean differences
##                     count
## Balanced, <0.05        50
## Not Balanced, >0.05     2
## 
## Variable with the greatest mean difference
##    Variable Max.Diff.Adj         M.Threshold
##  NU_IDADE_N       0.3693 Not Balanced, >0.05
## 
## Effective sample sizes
##                  no    preg puerp
## Unadjusted 32081.   2904.     715
## Adjusted    1062.01 1739.87   715
\end{verbatim}

\end{document}
